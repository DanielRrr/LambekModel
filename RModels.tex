\documentclass[a4paper]{article}
\usepackage{amsmath}
\usepackage{amsthm}
\usepackage{amsfonts}
\usepackage{amssymb}
\usepackage{bussproofs}
\usepackage{mathtools}
\usepackage{verbatim}
\usepackage{dsfont}
\usepackage{mathabx}
\usepackage[all, 2cell]{xy}
\usepackage[all]{xy}
\usepackage{wasysym}
\usepackage{rotating}
\usepackage{geometry}
\usepackage{trfsigns}
\usepackage{cmll}
\usepackage{authblk}

\theoremstyle{defin}
\newtheorem{defin}{Definition}

\theoremstyle{theorem}
\newtheorem{theorem}{Theorem}

\theoremstyle{prop}
\newtheorem{prop}{Proposition}

\theoremstyle{lemma}
\newtheorem{lemma}{Lemma}

\theoremstyle{ex}
\newtheorem{ex}{Example}

\theoremstyle{col}
\newtheorem{col}{Corollary}
\usepackage{listings} 		% for source code
\date{}
\author[1,2]{Daniel Rogozin}
\affil[1]{Lomonosov Moscow State University}
\affil[2]{Serokell O\"{U}}
\title{On $R$-models}

\begin{document}

\maketitle

\begin{defin}
  $ $

  \begin{prooftree}
  \AxiomC{$ $}
  \RightLabel{\scriptsize{ax}}
  \UnaryInfC{$A \rightarrow A$}
  \end{prooftree}

\begin{minipage}{0.5\textwidth}
  \begin{flushleft}
        \begin{prooftree}
      \AxiomC{$\Gamma \rightarrow A$}
      \AxiomC{$\Delta, B, \Theta \rightarrow C$}
      \RightLabel{$\backslash \rightarrow$}
      \BinaryInfC{$\Delta, \Gamma, A \backslash B, \Theta \rightarrow C$}
    \end{prooftree}

    \begin{prooftree}
      \AxiomC{$\Gamma \rightarrow A$}
      \AxiomC{$\Delta, B, \Theta \rightarrow C$}
      \RightLabel{$/ \rightarrow$}
      \BinaryInfC{$\Delta, B / A, \Gamma, \Theta \rightarrow C$}
    \end{prooftree}

    \begin{prooftree}
      \AxiomC{$\Gamma, A, B, \Delta \rightarrow C$}
      \RightLabel{$\bullet \rightarrow$}
      \UnaryInfC{$\Gamma, A \bullet B, \Delta \rightarrow C$}
    \end{prooftree}
\end{flushleft}
\end{minipage}
\begin{minipage}{0.5\textwidth}
  \begin{flushright}
       \begin{prooftree}
      \AxiomC{$A, \Pi \rightarrow B$}
      \RightLabel{$\rightarrow \backslash$}
      \UnaryInfC{$\Pi \rightarrow A \backslash B$}
    \end{prooftree}

    \begin{prooftree}
      \AxiomC{$\Pi, A \rightarrow B$}
      \RightLabel{$\rightarrow /$}
      \UnaryInfC{$\Pi \rightarrow B / A$}
    \end{prooftree}

    \begin{prooftree}
      \AxiomC{$\Gamma \rightarrow A$}
      \AxiomC{$\Delta \rightarrow B$}
      \RightLabel{$\rightarrow \bullet$}
      \BinaryInfC{$\Gamma, \Delta \rightarrow A \bullet B$}
    \end{prooftree}
\end{flushright}
\end{minipage}
\end{defin}

\begin{defin}
  $ $

  $R$-model is a pair $\mathcal{M} = \langle W, R, v \rangle$,
  where $R$ is a transitive relation on $W$ and $v : Tp \to 2^R$ is a valuation, such that:

  \begin{enumerate}
    \item $\mathcal{M}, w \models p_i \Leftrightarrow w \in v(p_i)$;
    \item $\mathcal{M}, \langle a, b \rangle \models A \bullet B \Leftrightarrow
      \text{there exists $c \in W$, $\mathcal{M}, a \models A$ and $\mathcal{M}, b \models B$}$;
    \item $\mathcal{M}, \langle a, b \rangle \models A \backslash B \Leftrightarrow
      \text{for all $c \in R^{-1}(a)$, $\mathcal{M}, \langle c, a \rangle \models A$ implies
      $\mathcal{M}, \langle c, b \rangle \models B$}$;
    \item $\mathcal{M}, \langle a, b \rangle \models B / A \Leftrightarrow
        \text{for all $c \in R(a)$, $\mathcal{M}, \langle a, c \rangle \models A$ implies
        $\mathcal{M}, \langle b, c \rangle \models B$}$;
    \item $\mathcal{M}, \langle a, b \rangle \models \Gamma \rightarrow A \Leftrightarrow
    \mathcal{M}, \langle a, b \rangle \models \Gamma \text{ implies } \mathcal{M}, \langle a, b \rangle \models A$
  \end{enumerate}
\end{defin}

where $\mathcal{M}, \langle a, b \rangle \models \Gamma$ denotes $\mathcal{M}, \langle a, b \rangle \models A_1 \bullet \dots \bullet A_n$, or,
equivalently, there exist $c_1, \dots, c_{n-1}$, such that $\mathcal{M}, \langle a, c_1 \rangle \models A_1,
\mathcal{M}, \langle c_1, c_2 \rangle \models A_2, \dots, \mathcal{M}, \langle c_{n-1}, b \rangle \models A_n$ implies that
$\mathcal{M}, \langle a, b \rangle \models B$.

\begin{theorem}
  Let $\mathbb{F}$ be a $R$-frame, then $\mathbb{F} \models L$.
\end{theorem}

\begin{defin}
  $ $

  Let $\mathcal{F}_1$, $\mathcal{F}_2$ be $R$-frames and $\mathcal{M}_1 = \langle \mathcal{F}_1, v_1 \rangle$, $\mathcal{M}_2 = \langle \mathcal{F}_2, v_2 \rangle$
  be $R$-models.

  A map $f : W_1 \to W_2$ is said to be a $R$-frame $p$-morphism if the following conditions hold:

  \begin{enumerate}
    \item $f$ is onto;
    \item $\forall a, b \in W_1 (a R_1 b \Rightarrow f(a) R_2 f (b))$ (monotonicity);
    \item $\forall d \in W_1, c \in W_2, f(d) R_2 c \Rightarrow \exists c' \in W_1, f(c') = c \: \& \: d R_1 c'$ (left lift property);
    \item $\forall d \in W_1, c \in W_2, c R_2 f(d) \Rightarrow \exists c' \in W_1, f(c') = c \: \& \: c' R_1 d$ (right lift property).
  \end{enumerate}

  A map $f : \mathcal{F}_1 \to \mathcal{F}_2$ is $R$-model $p$-moprhism, iff:

\begin{center}
 $\mathcal{M}_1, \langle a, b \rangle \models p_i \Leftrightarrow \mathcal{M}_2, \langle f(a), f(b) \rangle \models p_i$
\end{center}

\end{defin}

\begin{lemma}
  Let $f : \mathcal{M}_1 \twoheadrightarrow \mathcal{M}_2$, then
  $\mathcal{M}_1, \langle a, b \rangle \models A \Leftrightarrow \mathcal{M}_2, \langle f(a), f(b) \rangle \models A$,
  for all $a, b \in W_1$ and for all $A \in Fm$
\end{lemma}

\begin{proof}
$ $

  \begin{enumerate}
    \item $\Rightarrow$
      \begin{enumerate}
        \item Basic case: follows from the definition.
        \item Let $A \eqcirc B \bullet C$ and $\mathcal{M}_1, \langle a, b \rangle \models B \bullet C$, then
        there exists $c \in W_1$, such that $\mathcal{M}_1, \langle a, c \rangle \models B$ and
        $\mathcal{M}_1, \langle c, b \rangle \models C$.

        Then, $a R_1 c$ and $c R_1 b$, so $f(a) R_2 f(c)$ and $f(c) R_2 f(b)$.

        Thus, by IH, $\mathcal{M}_2, \langle f(a), f(c) \rangle \models B$ and
        $\mathcal{M}_2, \langle f(c), f(b) \rangle \models C$, so $\mathcal{M}_2, \langle f(a), f(b) \rangle \models B \bullet C$.
        \item Let $A \eqcirc B \backslash C$ and $\mathcal{M}_1, \langle a, b \rangle \models B \backslash C$.
        Let $c \in W_2$, such that $c R_2 f(a)$.

        Then, by left lift property, there exist $d \in W_1$, such that $f(d) = c$ and $d R_1 a$.

        Thus, $\mathcal{M}_1, \langle d, a \rangle \models A$ implies
        $\mathcal{M}_1, \langle d, b \rangle \models B$.

        By IH, $\mathcal{M}_2, \langle c, f(a) \rangle \models A$ implies
        $\mathcal{M}_2, \langle c, f(b) \rangle \models B$, then $\mathcal{M}_2, \langle f(a), f(b) \rangle \models A \backslash B$.

        \item Similarly to (c), but via right lift property.
      \end{enumerate}
    \item $\Leftarrow$
    \begin{enumerate}
      \item Basic case: follows from the definition.

      \item Let $A \eqcirc B \bullet C$.
      Let $\mathcal{M}_2, \langle f(a), f(b) \rangle \models B \bullet C$. Then
      there exists $c \in W_2$, such that $\mathcal{M}_2, \langle f(a), c \rangle \models B$ and
      $\mathcal{M}_2, \langle c, f(b) \rangle \models C$.

      So far as $f$ is onto, then there exists $d \in W_1$, such that $c = f(d)$, then $\mathcal{M}_2, \langle f(a), f(d) \rangle \models B$ and
      $\mathcal{M}_2, \langle f(d), f(b) \rangle \models C$, and, by IH, $\mathcal{M}_1, \langle a, d \rangle \models B$ and
      $\mathcal{M}_1, \langle d, b \rangle \models C$, then $\mathcal{M}_1, \langle a, b \rangle \models B \bullet C$.

      \item Let $A \eqcirc B \backslash C$ and $\mathcal{M}_2, \langle f(a), f(b) \rangle \models B \backslash C$.

      Let $c \in W_1$ and $c R_1 a$, then $f(c) R_1 f(a)$ by monotonicity, so $\mathcal{M}_2, \langle f(c), f(a) \rangle \models A$ implies
      $\mathcal{M}_2, \langle f(c), f(b) \rangle \models B$.

      By IH, $\mathcal{M}_1, \langle c, a \rangle \models A$ implies $\mathcal{M}_1, \langle c, b \rangle \models B$. Thus,
      $\mathcal{M}_1, \langle c, a \rangle \models A \backslash B$.
      \item Similarly to (c).
    \end{enumerate}
  \end{enumerate}
\end{proof}

\begin{lemma}
$ $

  \begin{enumerate}
    \item Let $\mathcal{M}_1$ and $\mathcal{M}_2$ be $R$-models and $f : \mathcal{M}_1 \twoheadrightarrow \mathcal{M}_2$. Then $\mathcal{M}_1 \models A$ iff
    $\mathcal{M}_2 \models A$.

    \item Let $\mathcal{F}_1$ and $\mathcal{F}_2$ be $R$-frames and $f : \mathcal{F}_1 \twoheadrightarrow \mathcal{F}_2$,
    then $\mathcal{F}_1 \models A$ implies $\mathcal{F}_2 \models A$.
  \end{enumerate}
\end{lemma}

\begin{proof}
$ $

\begin{enumerate}
  \item
  \begin{itemize}
    \item Only if:

    Let $\mathcal{M}_1 \models A$. Let $c, d \in W_2$. So far as $f$ is onto, then there exists $a, b \in W_1$, such that $f(a) = c$ and $f(b) = d$.

    Then $\mathcal{M}_1, \langle a, b \rangle \models A$, thus $\mathcal{M}_2 \langle f(a), f(b) \rangle \models A$.
    That is, $\mathcal{M}_2 \langle c, d \rangle \models A$

    \item If:

    Follows from the previous lemma.
  \end{itemize}
  \item Let $f : \mathcal{F}_1 \twoheadrightarrow \mathcal{F}_2$ and $\mathcal{F}_1 \models A$.

  Let $\mathcal{M}_1 = \langle \mathcal{F}_1, v_1 \rangle$ and $\mathcal{M}_2 = \langle \mathcal{F}_2, v_2 \rangle$, such that
  for all $p \in Tp$, $\mathcal{M}_1, \langle a, b \rangle \models p \Leftrightarrow_{def} \mathcal{M}_1, \langle f(a), f(b) \rangle \models p$.
  Thus, $\mathcal{M}_1 \twoheadrightarrow \mathcal{M}_2$ and $\mathcal{M}_1 \models A$. Thus, $\mathcal{M}_2 \models A$.
\end{enumerate}
\end{proof}

\begin{defin}
  $ $

\begin{enumerate}
  \item Let $\mathcal{F} = \langle W, R \rangle$ be a transitive frame and $V \subseteq W$, such that $V \neq \emptyset$ and $V$ is
  downward and upward closed with respect to $R$. Then a frame $\mathcal{F} \restriction V = \langle V, R \cap V \times V \rangle$ is a generated subframe.
  \item If, $\mathcal{M} = \langle W, R, \theta$ and $V \subseteq W$, such that $\mathcal{F} \restriction V$ is a genereted subframe. Then
  $\mathcal{M} \restriction V = \langle V, R \cap V \times V, \theta^{'}\rangle$ is a generated submodel, where $\theta^{'}(p) = \theta(p) \cap V$.
\end{enumerate}
\end{defin}

\begin{lemma}
  Let $\mathcal{M} = \langle W, R, \theta$ and $V$ is $R$-closed subset of $W$, then:
  \begin{enumerate}
    \item If $a, b \in V$, then $\mathcal{M}, \langle a, b \rangle \models A$ iff $\mathcal{M} \restriction V, \langle a, b \rangle \models A$;
    \item $Log(\mathcal{F}) \subseteq Log(\mathcal{F} \restriction V)$
  \end{enumerate}
\end{lemma}

\begin{proof}
  $ $
  \begin{enumerate}
    \item
    \begin{enumerate}
      \item Let $a, b \in V$ and $\langle a, b \rangle \in \theta(p)$, then $\mathcal{M} \restriction V, \langle a, b \rangle \models p$, for $p \in Tp$.
      \item Let $A \eqcirc B \bullet C$ and $\mathcal{M}, \langle a, b \rangle \models B \bullet C$, then
      there exists $c \in W$, such that $a R C$ and $\mathcal{M}, \langle a, c \rangle \models B$, $c R b$ and $\mathcal{M}, \langle c, b \rangle \models C$.
      So far as $V$ is upwardly and downwardly closed, then $c \in V$. Then, by IH, $\mathcal{M} \restriction V , \langle a, c \rangle \models B$, $c R b$ and
      $\mathcal{M} \restriction V, \langle c, b \rangle \models C$. So $\mathcal{M} \restriction V \langle a, b \rangle \models B \bullet C$.
      \item Let $A \eqcirc B \backslash C$ and $\mathcal{M}, \langle a, b \rangle \models B \backslash C$. Then for all $c \in R^{-1}(a)$, $\mathcal{M},
      \langle c, a \rangle \models B$ implies $\mathcal{M}, \langle c, b \rangle \models C$. Let $\mathcal{M}, \langle c, a \rangle \models B$.
      By IH, $\mathcal{M} \restriction V, \langle c, a \rangle \models B$.

      $c \in V$, because $V$ is $R$-closed upwardly and downloadly. On the other hand, $c R b$, so $b \in V$. Hence,
      $\langle c, b \rangle \in \mathcal{F} \restriction V$. By IH, $\mathcal{M} \restriction V, \langle c, b \rangle \models C$. Hence,
      $\mathcal{M} \restriction V, \langle a, b \rangle \models A \backslash B$.
    \end{enumerate}
    \item Let $\mathcal{F} \restriction V\not\models \Gamma \to A$. Then there exist a valuation $v$ and $\langle a, b \rangle \in R \cap V \times V$,
    such that $\mathcal{M} \restriction V, \langle a,b \rangle \models \Gamma$ and $\mathcal{M} \restriction V, \langle a,b \rangle\not\models A$.

    Let us define a valuation $v^{'}$ for the frame $\mathcal{F}$, such that $v(p) = v^{'}(p)$. Then, by the first part, $\mathcal{M}, \langle a, b \rangle \models \Gamma$ and $\mathcal{M}, \langle a, b\rangle\not\models A$. So, $\mathcal{F}\not\models \Gamma \to A$.
  \end{enumerate}
\end{proof}

\begin{col}
  $ $

  Let $\mathcal{F}$ be a frame and $\mathcal{F} \restriction V$ is a generated subframe of $\mathcal{F}$.
  Then $Log(\mathcal{F}) = Log(\mathcal{F} \restriction V)$.
\end{col}

Let $\mathbb{Z}^{*}$ be the set of all finite sequences of integers. Let us define the following relation $R$.

\end{document}
