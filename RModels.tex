\documentclass[a4paper]{article}
\usepackage{amsmath}
\usepackage{amsthm}
\usepackage{amsfonts}
\usepackage{amssymb}
\usepackage{bussproofs}
\usepackage{mathtools}
\usepackage{verbatim}
\usepackage{dsfont}
\usepackage{mathabx}
\usepackage[all, 2cell]{xy}
\usepackage[all]{xy}
\usepackage{wasysym}
\usepackage{rotating}
\usepackage{geometry}
\usepackage{trfsigns}
\usepackage{cmll}
\usepackage{authblk}

\theoremstyle{defin}
\newtheorem{defin}{Definition}

\theoremstyle{theorem}
\newtheorem{theorem}{Theorem}

\theoremstyle{prop}
\newtheorem{prop}{Proposition}

\theoremstyle{lemma}
\newtheorem{lemma}{Lemma}

\theoremstyle{ex}
\newtheorem{ex}{Example}

\theoremstyle{col}
\newtheorem{col}{Corollary}
\usepackage{listings} 		% for source code
\date{}
\author[1,2]{Daniel Rogozin}
\affil[1]{Lomonosov Moscow State University}
\affil[2]{Serokell O\"{U}}
\title{On $R$-models}

\begin{document}

\maketitle

\begin{defin}
  $ $

  $R$-model is a pair $\mathcal{M} = \langle W, R, v \rangle$,
  where $R$ is a transitive relation on $W$ and $v : Tp \to 2^R$ is a valuation, such that:

  \begin{enumerate}
    \item $\mathcal{M}, w \Vdash p_i \Leftrightarrow w \in v(p_i)$;
    \item $\mathcal{M}, \langle a, b \rangle \Vdash A \bullet B \Leftrightarrow
      \text{there exists $c \in W$, $\mathcal{M}, a \Vdash A$ and $\mathcal{M}, b \Vdash B$}$;
    \item $\mathcal{M}, \langle a, b \rangle \Vdash A \backslash B \Leftrightarrow
      \text{for all $c \in R^{-1}(a)$, $\mathcal{M}, \langle c, a \rangle \Vdash A$ implies
      $\mathcal{M}, \langle c, b \rangle \Vdash B$}$;
    \item $\mathcal{M}, \langle a, b \rangle \Vdash B / A \Leftrightarrow
        \text{for all $c \in R(a)$, $\mathcal{M}, \langle a, c \rangle \Vdash A$ implies
        $\mathcal{M}, \langle b, c \rangle \Vdash B$}$;
    \item $\mathcal{M}, \langle a, b \rangle \Vdash \Gamma \rightarrow A \Leftrightarrow
    \mathcal{M}, \langle a, b \rangle \Vdash \Gamma \text{ implies } \mathcal{M}, \langle a, b \rangle \Vdash A$
  \end{enumerate}
\end{defin}

where $\mathcal{M}, \langle a, b \rangle \Vdash \Gamma$ denotes $\mathcal{M}, \langle a, b \rangle \Vdash A_1 \bullet \dots \bullet A_n$, or,
equivalently, there exist $c_1, \dots, c_{n-1}$,
such that $\mathcal{M}, \langle a, c_1 \rangle \Vdash A_1,
\mathcal{M}, \langle c_1, c_2 \rangle \Vdash A_2, \dots,
\mathcal{M}, \langle c_{n-1}, b \rangle \Vdash A_n$ implies that $\mathcal{M}, \langle a, b \rangle \Vdash B$.

\begin{theorem}
  Let $\mathbb{F}$ is a $R$-frame, then $\mathbb{F} \models L$.
\end{theorem}

\begin{defin}
  $ $

  Let $\mathcal{F}_1$, $\mathcal{F}_2$ be $R$-frames and
  $\mathcal{M}_1 = \langle \mathcal{F}_1, v_1 \rangle$, $\mathcal{M}_2 = \langle \mathcal{F}_2, v_2 \rangle$
  be $R$-models.

  A map $f : W_1 \to W_2$ is said to be a $R$-frame $p$-morphism if the following conditions hold:

  \begin{enumerate}
    \item $f$ is onto;
    \item $\forall a, b \in W_1 (a R_1 b \Rightarrow f(a) R_2 f (b))$ (monotonicity);
    \item $\forall d \in W_1, c \in W_2, f(d) R_2 c \Rightarrow \exists c' \in W_1, f(c') = c \: \& \: d R_1 c'$ (left lift property);
    \item $\forall d \in W_1, c \in W_2, c R_2 f(d) \Rightarrow \exists c' \in W_1, f(c') = c \: \& \: c' R_1 d$ (right lift property);
  \end{enumerate}

  A map $f : \mathcal{F}_1 \to \mathcal{F}_2$ is $R$-model $p$-moprhism, iff:

  \begin{itemize}
    \item $\mathcal{M}_1, \langle a, b \rangle \Vdash p_i \Leftrightarrow \mathcal{M}_2, \langle f(a), f(b) \rangle \Vdash p_i$
  \end{itemize}
\end{defin}

\begin{lemma}
  Let $f : \mathcal{M}_1 \twoheadrightarrow \mathcal{M}_2$, then
  $\mathcal{M}_1, \langle a, b \rangle \Vdash A \Leftrightarrow \mathcal{M}_2, \langle f(a), f(b) \rangle \Vdash A$,
  for all $a, b \in W_1$ and for all $A \in Fm$
\end{lemma}

\begin{proof}
$ $

  \begin{enumerate}
    \item $\Rightarrow$
      \begin{enumerate}
        \item Basic case: follows from the definition.
        \item Let $A \eqcirc B \bullet C$ and $\mathcal{M}_1, \langle a, b \rangle \Vdash B \bullet C$, then
        there exists $c \in W_1$, such that $\mathcal{M}_1, \langle a, c \rangle \Vdash B$ and
        $\mathcal{M}_1, \langle c, b \rangle \Vdash C$.

        Then, $a R_1 c$ and $c R_1 b$, so $f(a) R_2 f(c)$ and $f(c) R_2 f(b)$.

        Thus, by IH, $\mathcal{M}_2, \langle f(a), f(c) \rangle \Vdash B$ and
        $\mathcal{M}_2, \langle f(c), f(b) \rangle \Vdash C$, so $\mathcal{M}_2, \langle f(a), f(b) \rangle \Vdash B \bullet C$.
        \item Let $A \eqcirc B \backslash C$ and $\mathcal{M}_1, \langle a, b \rangle \Vdash B \backslash C$.
        Let $c \in W_2$, such that $c R_2 f(a)$.

        Then, by left lift property, there exist $d \in W_1$, such that $f(d) = c$ and $d R_1 a$.

        Thus, $\mathcal{M}_1, \langle d, a \rangle \Vdash A$ implies
        $\mathcal{M}_1, \langle d, b \rangle \Vdash B$.

        By IH, $\mathcal{M}_2, \langle c, f(a) \rangle \Vdash A$ implies
        $\mathcal{M}_2, \langle c, f(b) \rangle \Vdash B$, then $\mathcal{M}_2, \langle f(a), f(b) \rangle \Vdash A \backslash B$.

        \item Similarly to (c), but via right lift property.
        \item Let $\mathcal{M}_1, \langle a, b \rangle \Vdash A_1, \dots, A_n \rightarrow A$.

        Thus, there exist $c_1, \dots, c_{n-1} \in W_1$, such that
        $\mathcal{M}_1, \langle a, c_1 \rangle \Vdash A_1, \dots, \mathcal{M}_1, \langle c_{n-2}, c_{n-1} \rangle \Vdash A_n$
        implies that $\mathcal{M}_1, \langle c_{n-1}, b \rangle \Vdash B$.

        By IH, $\mathcal{M}_2, \langle f(a), f(c_1) \rangle \Vdash A_1, \dots, \mathcal{M}_1, \langle f(c_{n-1}), f(b) \rangle \Vdash A_n$.

        So, $\mathcal{M}_2, \langle f(a), f(b) \rangle \Vdash A_1 \bullet \dots \bullet A_n$.

        Thus, $\mathcal{M}_2, \langle f(a), f(b) \rangle \Vdash B$.
      \end{enumerate}
    \item $\Leftarrow$
    \begin{enumerate}
      \item Basic case: follows from the definition.

      \item Let $A \eqcirc B \bullet C$.
      Let $\mathcal{M}_2, \langle f(a), f(b) \rangle \Vdash B \bullet C$. Then
      there exists $c \in W_2$, such that $\mathcal{M}_2, \langle f(a), c \rangle \Vdash B$ and
      $\mathcal{M}_2, \langle c, f(b) \rangle \Vdash C$.

      So far as $f$ is surjection, then there exists $d \in W_1$, such that
      $c = f(d)$, then $\mathcal{M}_2, \langle f(a), f(d) \rangle \Vdash B$ and
      $\mathcal{M}_2, \langle f(d), f(b) \rangle \Vdash C$, and, by IH,
      $\mathcal{M}_1, \langle a, d \rangle \Vdash B$ and $\mathcal{M}_1, \langle d, b \rangle \Vdash C$,
      then $\mathcal{M}_1, \langle a, b \rangle \Vdash B \bullet C$.

      \item Let $A \eqcirc B \backslash C$ and $\mathcal{M}_2, \langle f(a), f(b) \rangle \Vdash B \backslash C$.

      Let $c \in W_1$ and $c R_1 a$, then $f(c) R_1 f(a)$ by monotonicity,
      so $\mathcal{M}_2, \langle f(c), f(a) \rangle \Vdash A$ implies
      $\mathcal{M}_2, \langle f(c), f(b) \rangle \Vdash B$.

      By IH, $\mathcal{M}_1, \langle c, a \rangle \Vdash A$ implies
      $\mathcal{M}_1, \langle c, b \rangle \Vdash B$. Thus, $\mathcal{M}_1, \langle c, a \rangle \Vdash A \backslash B$.
      \item Similarly to (c).
      \item Let $\mathcal{M}_2, \langle f(a), f(b) \rangle \Vdash \Gamma \rightarrow A$, so
      by case (b) and IH, $\mathcal{M}_2, \langle a, b \rangle \Gamma$, so $\mathcal{M_1}, \langle a, b \rangle A$.
    \end{enumerate}
  \end{enumerate}
\end{proof}

\begin{lemma}
$ $

  \begin{enumerate}
    \item Let $\mathcal{M}_1$ and $\mathcal{M}_2$ be $R$-models and $\mathcal{M}_1 \twoheadrightarrow \mathcal{M}_2$. Then $\mathcal{M}_1 \models A$ iff $\mathcal{M}_2 \models A$.

    \item Let $\mathcal{F}_1$ and $\mathcal{F}_2$ be $R$-frames and $\mathcal{F}_1 \twoheadrightarrow \mathcal{F}_2$,
    then $\mathcal{F}_1 \models A$ implies $\mathcal{F}_2 \models A$.

    \item $\mathcal{F}_1 \cong \mathcal{F}_2$, then $Log(\mathcal{F}_1) = Log(\mathcal{F}_2)$.
  \end{enumerate}
\end{lemma}

\begin{proof}
  Standardly.
\end{proof}

\end{document}
