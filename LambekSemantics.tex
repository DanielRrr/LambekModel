\documentclass[a4paper]{article}
\usepackage{amsmath}
\usepackage{amsthm}
\usepackage{amsfonts}
\usepackage{amssymb}
\usepackage{bussproofs}
\usepackage{mathtools}
\usepackage{verbatim}
\usepackage{dsfont}
\usepackage{mathabx}
\usepackage[all, 2cell]{xy}
\usepackage[all]{xy}
\usepackage{wasysym}
\usepackage{rotating}
\usepackage{geometry}
\usepackage{trfsigns}
\usepackage{cmll}
\newtheorem{theorem}{Theorem}
\newtheorem{prop}{Proposition}
\newtheorem{lemma}{Lemma}
\newtheorem{defin}{Definition}
\newtheorem{ex}{Example}
\newtheorem{col}{Corollary}
\newtheorem{con}{Consequence}
\usepackage{listings} 		% for source code
\date{}
\title{Categorical model of noncommutative linear logic with subexponentials}
\begin{document}


\maketitle

\begin{defin} A subexponential signature is an ordered quintuple:

  $\Sigma = \langle I, \preceq, W, C, E \rangle$,
\end{defin}

where $I = \{ s_1, \dots, s_n\}$, $\langle I, \preceq \rangle$ is a preorder.
$W, C, E$ are subsets of $I$ and $W \cup C \subseteq E$.

\begin{defin} Noncommutative linear logic with subexponentials ($SMALC_{\Sigma}$), where $\Sigma$ is a subexponential signature.

\begin{prooftree}
\AxiomC{$ $}
\RightLabel{\scriptsize{ax}}
\UnaryInfC{$A \Rightarrow A$}
\end{prooftree}

\begin{minipage}{0.5\textwidth}
  \begin{flushleft}
    \begin{prooftree}
      \AxiomC{$\Gamma \Rightarrow A$}
      \AxiomC{$\Delta, B, \Theta \Rightarrow C$}
      \RightLabel{$\backslash \rightarrow$}
      \BinaryInfC{$\Delta, \Gamma, A \backslash B, \Theta \Rightarrow C$}
    \end{prooftree}

    \begin{prooftree}
      \AxiomC{$\Gamma \Rightarrow A$}
      \AxiomC{$\Delta, B, \Theta \Rightarrow C$}
      \RightLabel{$/ \rightarrow$}
      \BinaryInfC{$\Delta, B / A, \Gamma, \Theta \Rightarrow C$}
    \end{prooftree}

    \begin{prooftree}
      \AxiomC{$\Gamma, A, B, \Delta \Rightarrow C$}
      \RightLabel{$\bullet \rightarrow$}
      \UnaryInfC{$\Gamma, A \bullet B, \Delta \Rightarrow C$}
    \end{prooftree}

    \begin{prooftree}
      \AxiomC{$\Gamma, A_i, \Delta \Rightarrow B$}
      \RightLabel{$\&, i = 1,2 \rightarrow$}
      \UnaryInfC{$\Gamma, A_1 \& A_2, \Delta \Rightarrow B$}
    \end{prooftree}

    \begin{prooftree}
      \AxiomC{$\Gamma, A, \Delta \Rightarrow C$}
      \AxiomC{$\Gamma, B, \Delta \Rightarrow C$}
      \RightLabel{$\vee \rightarrow$}
      \BinaryInfC{$\Gamma, A \vee B, \Delta \Rightarrow C$}
    \end{prooftree}

    \begin{prooftree}
      \AxiomC{$\Gamma, \Delta \Rightarrow A$}
      \RightLabel{${\bf 1} \rightarrow$}
      \UnaryInfC{$\Gamma, {\bf 1}, \Delta \Rightarrow A$}
    \end{prooftree}

    \begin{prooftree}
    \AxiomC{$\Gamma, A, \Delta \Rightarrow C$}
    \RightLabel{$! \rightarrow$}
    \UnaryInfC{$\Gamma, !^{s} A, \Delta \Rightarrow C$}
    \end{prooftree}
  \end{flushleft}
\end{minipage}
\begin{minipage}{0.5\textwidth}
  \begin{flushright}
    \begin{prooftree}
      \AxiomC{$A, \Pi \Rightarrow B$}
      \RightLabel{$\rightarrow \backslash$}
      \UnaryInfC{$\Pi \Rightarrow A \backslash B$}
    \end{prooftree}

    \begin{prooftree}
      \AxiomC{$\Pi, A \Rightarrow B$}
      \RightLabel{$\rightarrow /$}
      \UnaryInfC{$\Pi \Rightarrow B / A$}
    \end{prooftree}

    \begin{prooftree}
      \AxiomC{$\Gamma \Rightarrow A$}
      \AxiomC{$\Delta \Rightarrow B$}
      \RightLabel{$\rightarrow \bullet$}
      \BinaryInfC{$\Gamma, \Delta \Rightarrow A \bullet B$}
    \end{prooftree}

    \begin{prooftree}
      \AxiomC{$\Gamma \Rightarrow A$}
      \AxiomC{$\Gamma \Rightarrow B$}
      \RightLabel{$\rightarrow \&$}
      \BinaryInfC{$\Gamma \Rightarrow A \& B$}
    \end{prooftree}

    \begin{prooftree}
      \AxiomC{$\Gamma \Rightarrow A_i$}
      \RightLabel{$\rightarrow \vee, i = 1,2$}
      \UnaryInfC{$\Gamma \Rightarrow A_1 \vee A_2$}
    \end{prooftree}

    \begin{prooftree}
      \AxiomC{$ $}
      \RightLabel{$\rightarrow {\bf 1}$}
      \UnaryInfC{$\Rightarrow {\bf 1}$}
    \end{prooftree}

    \begin{prooftree}
    \AxiomC{$!^{s_1} A_1, \dots, !^{s_n} A_n \Rightarrow A$}
    \RightLabel{$\rightarrow !, \forall j, s_j \succeq s$}
    \UnaryInfC{$!^{s_1} A_1, \dots, !^{s_n} A_n \Rightarrow !^{s} A$}
    \end{prooftree}
  \end{flushright}
\end{minipage}

\begin{prooftree}
  \AxiomC{$\Gamma, \Delta \Rightarrow B$}
  \RightLabel{${\bf weak}_!, s \in C$}
  \UnaryInfC{$\Gamma, !^{s} A, \Delta \Rightarrow B$}
\end{prooftree}

    \begin{prooftree}
    \AxiomC{$\Gamma, !^{s} A, \Delta, !^{s} A, \Theta \Rightarrow B$}
    \RightLabel{${\bf ncontr}_1, s \in C$}
    \UnaryInfC{$\Gamma, !^{s} A, \Delta, \Theta \Rightarrow B$}
    \end{prooftree}

    \begin{prooftree}
      \AxiomC{$\Gamma, !^{s} A, \Delta, !^{s} A, \Theta \Rightarrow B$}
      \RightLabel{${\bf ncontr}_2, s \in C$}
      \UnaryInfC{$\Gamma, \Delta, !^{s} A, \Theta \Rightarrow B$}
    \end{prooftree}

    \begin{prooftree}
    \AxiomC{$\Gamma, \Delta, !^{s} A, \Theta \Rightarrow B$}
    \RightLabel{${\bf ex}_1, s \in E$}
    \UnaryInfC{$\Gamma, !^{s} A, \Delta, \Theta \Rightarrow A$}
    \end{prooftree}

    \begin{prooftree}
      \AxiomC{$\Gamma, !^{s} A, \Delta, \Theta \Rightarrow B$}
      \RightLabel{${\bf ex}_1, s \in E$}
      \UnaryInfC{$\Gamma, \Delta, !^{s} A, \Theta \Rightarrow A$}
    \end{prooftree}
\end{defin}

\begin{lemma}
  Let $A \Leftrightarrow B$, then $C [p_i := A] \Leftrightarrow C [p_i := B]$
\end{lemma}

\begin{proof}
  By induction on $C$.
\end{proof}

\begin{defin} Monoidal comonad
  $ $

  A monoidal comonad on some monoidal category $\mathcal{C}$ is a triple $\langle \mathcal{F}, \epsilon, \delta \rangle$,
  where $\mathcal{F}$ is a monoidal endofunctor and $\epsilon : \mathcal{F} \Rightarrow Id_{\mathcal{C}}$ (counit) and $\epsilon : \mathcal{F} \Rightarrow \mathcal{F}^2$ (comultiplication),
  such that the following diagrams commute:

\begin{minipage}{0.45\textwidth}
\begin{small}
\xymatrix{
\mathcal{F}A \otimes \mathcal{F}B \ar[dd]_{\delta_A \otimes \delta_B} \ar[r]^{\phi_{A,B}} & \mathcal{F}(A \otimes B) \ar[dr]^{\delta_{A \otimes B}} \\
&& \mathcal{F}\mathcal{F}(A \otimes B) \\
\mathcal{F}\mathcal{F}A \otimes \mathcal{F}\mathcal{F}B \ar[r]_{\phi_{\mathcal{F}A, \mathcal{F}B}} & \mathcal{F}(\mathcal{F}A \otimes \mathcal{F}B) \ar[ur]_{\mathcal{F}(\phi_{A,B})}
}
\end{small}
\end{minipage}%
\hfill
\begin{minipage}{0.5\textwidth}
\begin{tabular}{p{\textwidth}}
\xymatrix{
\mathcal{F}A \otimes \mathcal{F}B \ar[rr]^{\phi_{A,B}} \ar[dr]_{\epsilon_A \otimes \epsilon_B} && \mathcal{F}(A \otimes B) \ar[dl]^{\epsilon_{A \otimes B}} \\
& A \otimes B &
}
\end{tabular}
\end{minipage}%

\xymatrix{
&&&&&&& \mathds{1} \ar[r]^{\phi} \ar[d]_{\phi} & \mathcal{F}\mathds{1} \ar[d]^{\delta_{\mathds{1}}} \\
&&&&&&& \mathcal{F}\mathds{1} \ar[r]_{\mathcal{F}(\phi)} & \mathcal{F}\mathcal{F}\mathds{1}
}

\xymatrix{
&&&&&&& \mathds{1} \ar[dr]_{\phi} \ar[rr]^{id_{\mathds{1}}} && \mathds{1}\\
&&&&&&&& \mathcal{F}\mathds{1} \ar[ur]_{\epsilon_{\mathds{1}}} &
}
\end{defin}

\begin{defin} Biclosed monoidal category

  Let $\mathcal{C}$ be a monoidal category. Biclosed monoidal category is a monoidal category with the following additional data:
  \begin{enumerate}
    \item Bifunctors $\underline{\quad} \multimapinv \underline{\quad}, \underline{\quad} \multimap \underline{\quad} : \mathcal{C}^{op} \times \mathcal{C} \to \mathcal{C}$;
    \item Natural isomophism ${\bf curry}_{A,B,C} : Hom (A \otimes B, C) \cong (B, A \multimap C)$;
    \item Natural isomophism ${\bf curry}^{'}_{A,B,C} : Hom (A \otimes B, C) \cong (A, C \multimap B)$;
    \item For each $A, B \in Ob_{\mathcal{C}}$, there are exist arrows $ev_{A,B} : A \otimes (A \Rightarrow B) \rightarrow B$ and $ev_{A,B}^{'} : (B \Leftarrow A) \otimes A \rightarrow B$,
    such that for all $f : A \otimes C \rightarrow B$:
      \begin{enumerate}
        \item $\Lambda_l \circ (id_A \otimes {\bf curry}(f)) = f$;
        \item $\Lambda_r \circ ({\bf curry}^{'}(f) \otimes id_A) = f$
      \end{enumerate}
  \end{enumerate}
\end{defin}

\begin{defin}
  Let $F$ be endofunctor and $A \in Ob\mathcal{C}$, then a coalgebra of $F$ is a tuple $\langle A, \theta \rangle$, where $\theta : A \to F A$.
\end{defin}

Given coalgebras $\langle A, \theta \rangle$ and $\langle A, \psi \rangle$, a homomorphism is a morphism $f : A \to B$, s.t. the diagram below commutes:

\xymatrix{
&&&&&& A \ar[dd]_{f} \ar[rr]^{\theta} && F A \ar[dd]^{F f} \\
\\
&&&&&& B \ar[rr]_{\psi} && F B
}
that is, $F f \circ \theta = \psi \circ f$

\begin{defin} Subexponential model structure

  Let $\Sigma = \langle I, \preceq, W, C, E \rangle$ be a subexponential signature and $\mathcal{C}$ be a biclosed monoidal category,
  then a subexponential model structure is $\langle \mathcal{C}, \{ \mathcal{F}_{s} \}_{s \in I} \rangle$ with the following additional data:

\begin{itemize}
  \item for all $s \in I$, $\mathcal{F}_s$ is a monoidal comonad;
  \item if $s \in W$, then for all $A \in Ob(\mathcal{C})$, there exists a morphism ${w_{A}}_{s} : F_s A \to \mathds{1}$;
  \item if $s \in C$, then for all $A \in Ob(\mathcal{C})$, there exists morphisms
${w_{A}}_{l} : F_s A \otimes A \otimes F_s A \to F_s A \otimes B$ and ${w_{A}}_{r} : F_s A \otimes A \otimes F_s A \to B \otimes F_s A$;
  \item if $s \in E$, then for all $A \in Ob(\mathcal{C})$, there is an isomorpism, ${e_A} : F_s A \otimes B \cong B \otimes F_s A$;
  \item if $s_1 \in W$, $s_2 \in I$ and $s_1 \preceq s_2$, then there is a morphism ${w_{A}}_{s_2} : F_{s_2} A \to \mathds{1}$ for all $A \in Ob(\mathcal{C})$ and ditto for $E$ and $C$;
  \item Let $\bigotimes_{s \in J, i = 0}^{n} F_{s} A$, where $J \subset I$, and $s^{'} \in I$, s.t. $s \succeq s^{'}$ for all $s \in I'$;
  Then there exists morphism a morphism $\theta_{\bigotimes_{s \in J, i = 1}^{n} {F_{s}}_j A_i} : \bigotimes_{s \in J, i = 0}^{n} F_{s} A \to F_{s^{'}} (\bigotimes_{s \in J, i = 0}^{n} F_{s} A)$, such that
  $\langle \bigotimes_{s \in J, i = 1}^{n} {F_{s}}_j A_i, \theta_{\bigotimes_{s \in J, i = 1}^{n} {F_{s}}_j A_i} \rangle$ is a coalgebra on $F_s$.
\end{itemize}
\end{defin}

\begin{defin}
  Let $\langle \mathcal{C}, \{ \mathcal{F}_{s} \}_{s \in I} \rangle$ be a subexponential model structure
  for subexponential signature $\Sigma = \langle I, \preceq, W, C, E \rangle$.
  Let $v : Tp \to Ob(\mathcal{C})$ be a valuation map. Then the interpretation function $[\![.]\!]$ is defined as follows:

$\begin{array}{lll}
(1) &[\![{\bf 1}]\!] = \mathds{1}& \\
(2) &[\![A \backslash B]\!] = [\![A]\!] \multimap [\![B]\!]& \\
(3) &[\![A / B]\!] = [\![A]\!] \multimapinv [\![B]\!]& \\
(4) &[\![A \bullet B]\!] = [\![A]\!] \otimes [\![B]\!]&\\
(5) &[\![!_{s} A]\!] = F_s [\![A]\!]& \\
\end{array}$
\end{defin}

\begin{theorem} The following statements are equivalent:

  \begin{itemize}
    \item $SMLC_{\Sigma} + (cut) \vdash \Gamma \Rightarrow A$
    \item $SMLC_{\Sigma} \vdash \Gamma \Rightarrow A$
    \item $\exists f, f : [\![\Gamma]\!] \rightarrow [\![A]\!]$
  \end{itemize}
\end{theorem}

\begin{proof}
  \begin{itemize}
$ $

    \item (1) $\Rightarrow$ (2): cut elimination.
    \item (2) $\Rightarrow$ (3): Soundness:

\begin{prooftree}
  \AxiomC{}
  \UnaryInfC{$id_{A} : A \to A$}
\end{prooftree}

\begin{prooftree}
  \AxiomC{$f : \Gamma \rightarrow A$}
  \AxiomC{$g : \Delta \otimes B \otimes \Theta \rightarrow C$}
  \BinaryInfC{$g \circ (id_{\Delta} \otimes ({ev_{A,B}}_l \circ (f \otimes id_{A \multimap B})) \otimes id_{\Theta}): \Delta \otimes (\Gamma \otimes A \multimap B) \otimes \Theta  \rightarrow C$}
\end{prooftree}

\begin{prooftree}
  \AxiomC{$f : A \otimes \Pi \rightarrow B$}
  \UnaryInfC{$\Lambda_l(f) : \Pi \rightarrow A \multimap B$}
\end{prooftree}

\begin{prooftree}
  \AxiomC{$f : \Gamma \rightarrow A$}
  \AxiomC{$g : \Delta \otimes B \otimes \Theta \rightarrow C$}
  \BinaryInfC{$g \circ (id_{\Delta} \otimes ({ev_{A,B}}_l \circ (id_{B \multimapinv A} \otimes f)) \otimes id_{\Theta}) : \Delta \otimes (B \multimapinv A \otimes \Gamma) \otimes \Theta \rightarrow C$}
\end{prooftree}

\begin{prooftree}
  \AxiomC{$f : \Pi \otimes A \rightarrow B$}
  \UnaryInfC{$\Lambda_r(f) : \Pi \rightarrow B \multimapinv A$}
\end{prooftree}

\begin{prooftree}
  \AxiomC{$f : \Gamma \otimes A \otimes B \otimes \Delta \rightarrow C$}
  \UnaryInfC{$f \circ (\alpha_{\Gamma, A, B} \otimes id_{\Delta}) : \Gamma \otimes (A \otimes B) \otimes \Delta \rightarrow C$}
\end{prooftree}

\begin{prooftree}
  \AxiomC{$f : \Gamma \rightarrow A$}
  \AxiomC{$g : \Delta \rightarrow B$}
  \BinaryInfC{$f \otimes g : \Gamma \otimes \Delta \rightarrow A \otimes B$}
\end{prooftree}

\begin{prooftree}
  \AxiomC{$f : \Gamma \otimes A_i \otimes \Delta \rightarrow B$}
  \UnaryInfC{$f \circ (id_{\Gamma} \otimes \pi_i id_{\Delta}) : \Gamma \otimes (A_1 \times A_2) \otimes \Delta \rightarrow B$}
\end{prooftree}

\begin{prooftree}
  \AxiomC{$f : \Gamma \rightarrow A$}
  \AxiomC{$g : \Gamma \rightarrow B$}
  \BinaryInfC{$\langle f , g \rangle : \Gamma \rightarrow A \times B$}
\end{prooftree}

\begin{prooftree}
  \AxiomC{$f : \Gamma \otimes A \otimes \Delta \rightarrow C$}
  \AxiomC{$g : \Gamma \otimes B \otimes \Delta  \rightarrow C$}
  \BinaryInfC{$id_{\Gamma} \otimes [f,g] \otimes id_{\Delta} : \Gamma \otimes (A + B) \otimes \Delta \rightarrow C$}
\end{prooftree}

\begin{prooftree}
  \AxiomC{$ $}
  \UnaryInfC{$id_{\mathds{1}} : \mathds{1} \rightarrow \mathds{1}$}
\end{prooftree}

\begin{prooftree}
  \AxiomC{$f : \Gamma \otimes \Delta \rightarrow A$}
  \UnaryInfC{$f \circ (\rho_{\Gamma} \otimes id_{\Delta}) : (\Gamma \otimes \mathds{1}) \otimes \Delta \rightarrow A$}
\end{prooftree}

\begin{prooftree}
  \AxiomC{$f : \Gamma \otimes A \otimes \Delta \rightarrow B$}
  \UnaryInfC{$f \circ (id_{\Gamma} \otimes \delta_s^{A} \otimes id_{\Delta}) : \Gamma \otimes F_s A \otimes \Delta \rightarrow B$}
\end{prooftree}

\begin{prooftree}
  \AxiomC{$f : F_{s_1} A_1 \otimes \dots \otimes F_{s_n} A_n \rightarrow B$}
  \UnaryInfC{$F_s(f) : F_s(F_{s_1} A_1 \otimes \dots \otimes F_{s_n} A_n) \rightarrow F_s B$}
  \UnaryInfC{$F_s (f) \circ \theta_{\bigotimes_{s \in J, i = 1}^{n} {F_{s}}_j A_i} : F_{s_1} A_1 \otimes \dots \otimes F_{s_n} A_n \rightarrow F_s B$}
\end{prooftree}

\begin{prooftree}
  \AxiomC{$f : \Gamma \otimes \Delta \rightarrow A$}
  \UnaryInfC{$f \circ (\rho_{\Gamma} \otimes id_{\Delta}) : (\Gamma \otimes \mathds{1}) \otimes \Delta \rightarrow A$}
  \UnaryInfC{$f \circ (\rho_{\Gamma} \otimes id_{\Delta}) \circ (id_{\Gamma} \otimes {w_{A}}_{s}) \otimes id_{\Delta} : (\Gamma \otimes F_s A) \otimes \Delta \rightarrow A$}
\end{prooftree}

\begin{prooftree}
  \AxiomC{$f : \Gamma \otimes (F_s A \otimes B \otimes F_s A) \otimes \Delta \rightarrow C$}
  \UnaryInfC{$f \circ (id_{\Gamma} \otimes {c_A}_{s}^l \otimes id_{\Delta}) : \Gamma \otimes (F_s A \otimes B) \otimes \Delta \rightarrow C$}
\end{prooftree}

\begin{prooftree}
  \AxiomC{$f : \Gamma \otimes (F_s A \otimes B \otimes F_s A) \otimes \Delta \rightarrow C$}
  \UnaryInfC{$(id_{\Gamma} \otimes {c_A}_{s}^r \otimes id_{\Delta}) \circ f: \Gamma \otimes (B \otimes F_s A) \otimes \Delta \rightarrow C$}
\end{prooftree}

\begin{prooftree}
  \AxiomC{$f : \Gamma \otimes (\Delta \otimes F_s A) \otimes \Theta \rightarrow B$}
  \UnaryInfC{$(id_{\Gamma} \otimes (id_{\Delta} \otimes {e_A}_{s}) \otimes id_{\Theta}) \circ f: \Gamma \otimes (F_s A \otimes \Delta) \otimes \Theta \rightarrow B$}
\end{prooftree}

\begin{prooftree}
  \AxiomC{$f : \Gamma \otimes (F_s A \otimes \Delta) \otimes \Theta \rightarrow B$}
  \UnaryInfC{$(id_{\Gamma} \otimes (id_{\Delta} \otimes {e_A}_{s}^{-1}) \otimes id_{\Theta}) \circ f: \Gamma \otimes (\Delta \otimes F_s A) \otimes \Theta \rightarrow B$}
\end{prooftree}

    \item Completeness:

\begin{defin}
\end{defin}

  \end{itemize}
\end{proof}

\section{Concrete model}

\begin{defin} Quantale
  A quantale is a triple $\langle A, \bigvee, \cdot \rangle$, such that $\langle A, \bigvee \rangle$
is a complete lattice and $\langle A, \cdot \rangle$ is a semigroup. A quanlate is called unital, if $\langle A, \cdot \rangle$
is a monoid.
\end{defin}

It is easy to see, that any (unital) quantale is a residual (monoid) semigroup. We define divisions as follows:

\begin{enumerate}
\item $a \backslash b = \bigvee \{ c \: | \: a \cdot c \leq b \}$
\item $b / a = \bigvee \{ c \: | \: c \cdot a \leq b \}$
\end{enumerate}

\begin{defin}
  Let $\langle A, \bigvee, \cdot \rangle$ be a quantale. The center of a quantale is the set $Z(Q) = \{ a \in Q \: | \: \forall b \in Q, a \cdot b = b \cdot a \}$
\end{defin}

\begin{defin} An open modality on quantale $Q$ is a map $I : Q \to Q$, such that

\begin{enumerate}
  \item $I(x) \leq x$;
  \item $I(x) = I(I(x))$;
  \item $x \leq y \Rightarrow I(x) \leq I(y)$;
  \item $I(x) \cdot I(y) = I(I(x) \cdot I(y))$.
\end{enumerate}
\end{defin}

\begin{lemma}
$ $

  Let $\langle A, \bigvee, \cdot \rangle$ be a quantale and $I : Q \to Q$ is an open modality on $Q$, then
  $I(x) \cdot I(y) \leq I(x \cdot y)$.
\end{lemma}

\begin{proof}
$ $

  $I(x) \cdot I(y) \leq x \cdot y$, then $I(I(x) \cdot I(y)) \leq I(x \cdot y)$, but
$I(x) \cdot I(y) \leq I(I(x) \cdot I(y))$. Thus, $I(x) \cdot I(y) \leq I(x \cdot y)$.
\end{proof}

\begin{defin}
  An open modality is called central, if $\forall a, b \in Q, I(a) \cdot b = b \cdot I(a)$.
\end{defin}

\begin{defin}
  An open modality is called weak idempotent, if $\forall a, b \in Q, I(a) \cdot b \leq I(a) \cdot b \cdot I(a)$ and
  $b \cdot I(a) \leq I(a) \cdot b \cdot I(a)$.
\end{defin}

\begin{defin}
  An open modality is called unital, if $\forall a \in Q, I(a) \leq e$.
\end{defin}

\begin{lemma}
  Let $I$ be an interior on some unital quantale $\langle Q, \bigvee, \cdot, e \rangle$.
  Then, if $I$ is unital and weak idempotent, then $I$ is central.
\end{lemma}

\begin{proof}
$ $

  $\begin{array}{lll}
  & b \cdot I(a) \leq & \\
  & \:\:\:\: \text{Right weak idempotence}& \\
  &I(a) \cdot b \cdot I(a) \leq & \\
  & \:\:\:\: \text{Unitality}& \\
  & I(a) \cdot b \cdot I(e) \leq & \\
  & \:\:\:\: \text{Identity}& \\
  &I(a) \cdot b \leq & \\
  & \:\:\:\: \text{Left weak idempotence}& \\
  &I(a) \cdot b \cdot I(a) \leq & \\
  & \:\:\:\: \text{Unitality}& \\
  &e \cdot b \cdot I(a) \leq & \\
  & \:\:\:\: \text{Identity}& \\
  &b \cdot I(a)&
  \end{array}$

Hence, $b \cdot I(a) = I(a) \cdot b$

\end{proof}

\begin{prop}
$ $

  Let $Q$ be a quantale and $S \subseteq Q$ a subquantale, then $I : Q \to Q$, such that
$I(a) = \bigvee \{ s \in S \: | \: x \leq a \}$, is an open modality.
\end{prop}

\begin{proof}
  See
\end{proof}

\begin{prop}
$ $

  Let $Q$ be a quantale and $S_1, S_2 \subseteq Q$, such that $S_1 \subseteq S_2$.

  Then $I_1 (a) \leq I_2 (a)$.
\end{prop}

\begin{proof}
$ $

  Let $a \in Q$, so $\{ s \in S_1 \: | \: s \leq a \} \subseteq \{ s \in S_2 \: | \: s \leq a \}$, so
  $\bigvee \{ s \in S_1 \: | \: s \leq a \} \subseteq \bigvee \{ s \in S_2 \: | \: s \leq a \}$.
  Thus, $I_1 (a) \leq I_2 (a)$.
\end{proof}

\begin{prop}
$ $

Let $Q$ be a quantale and $S \subseteq Q$ a subquantale, then the following operations are open modalities:

\begin{enumerate}
  \item $I_z (a) = \bigvee \{ s \in S \: | s \leq a, s \in Z(Q) \}$;
  \item $I_{\mathds{1}} (a) = \bigvee \{ s \in S \: | s \leq a, s \leq \mathds{1} \}$;
  \item $I_{idem} (a) = \bigvee \{ s \in S \: | s \leq a, \forall b \in Q, b \cdot s \vee s \cdot b \leq s \cdot b \cdot s\}$;
  \item $I_{z, \mathds{1}}, I_{z, idem}, I_{\mathds{1}, idem}, I_{z, \mathds{1}, idem}$.
\end{enumerate}
\end{prop}

\begin{proof}
  Immediatly.
\end{proof}

\begin{prop}
$ $

\begin{enumerate}
  \item $\forall a \in Q, I_{\mathds{1}, idem}(a) \leq I_z (a)$.
  \item $\forall a \in Q, I_{z, \mathds{1}, idem} = I_{\mathds{1}, idem}(a)$
\end{enumerate}

\end{prop}

\begin{proof}
  Follows from Lemma 3.
\end{proof}

\begin{prop}
$ $

\begin{enumerate}
  \item $I_z (a) \vee I_{\mathds{1}} (a) \vee I_{idem} (a) \leq I(a)$
  \item $I_{z, \mathds{1}, idem} \leq I_{z, \mathds{1}} (a) \wedge I_{z, idem} (a)$
\end{enumerate}
\end{prop}

\begin{prop}
  $ $

$I_2 (I_1 (a)) \leq I_2 (a)$.
\end{prop}

\begin{proof}
  $I_1 (a) \leq a \Rightarrow I_2 (I_1 (a)) \leq I_2 (a)$.
\end{proof}

\begin{prop}
$ $

  $I_2 (I_1 (a)) = I_1 (I_2 (a))$, for $I_2 (a) \leq I_1 (a)$.
\end{prop}

\begin{proof}
$ $

\begin{enumerate}
  \item $I_2 (I_1 (a)) \leq I_1 (I_2 (a))$

  $I_2 (I_1 (a)) \leq I_2 (a) = I_2 (I_2 (a)) \leq I_1 (I_2 (a))$
  \item $I_1 (I_2 (a)) \leq I_2 (I_1 (a))$

  $\begin{array}{lll}
  (1) & I_2 (a) \leq a& \\
  & \:\:\:\: \text{Monotonicity}& \\
  (2) & I_1 (I_2 (a)) \leq I_1 (a)& \\
  & \:\:\:\: \text{Monotonicity}& \\
  (3) & I_2 (I_1 (I_2 (a))) \leq I_2 (I_1 (a))& \\
  & \:\:\:\: \text{The previous part}& \\
  (4) & I_2 (I_1 (a)) \leq I_1 (I_2 (a)) & \\
  & \:\:\:\: \text{Monotonicity}& \\
  (5) & I_2 (I_2 (I_1 (a))) \leq I_2 (I_1 (I_2 (a))) & \\
  & \:\:\:\: \text{Idempotence}& \\
  (6) & I_1 (a) \leq I_2 (I_1 (I_2 (a))) & \\
  & \:\:\:\: \text{(1), (6), transitivity} & \\
  (7) & I_1 (I_2 (a)) \leq I_2 (I_1 (I_2 (a))) & \\
  & \:\:\:\: \text{(1), (6), transitivity} & \\
  (8) & I_2 (I_1 (I_2 (a))) \leq I_2 (I_1 (a))& \\
  & \:\:\:\: \text{(1), monotonicity twice} & \\
  (9) &I_1 (I_2 (a)) \leq I_2 (I_1 (a))& \\
  & \:\:\:\: \text{(7), (8), transitivity} & \\
  \end{array}$
\end{enumerate}
\end{proof}

\begin{lemma}
  $\forall a \in Q, I_1 (a) \leq I_2 (I_1 (a))$, if $I_2 (a) \leq I_1 (a)$.
\end{lemma}

\begin{proof}
  $\begin{array}{lll}
  &I_1 (a) \leq I_1 (a)&\\
  &I_2 (I_1 (a)) \leq I_1 (I_2 (a))&\\
  \end{array}$

\end{proof}

\begin{lemma}
  $I_1(a_1) \cdot I_2(a_2) \leq I^{'} (I_1(a_1) \cdot I_2(a_2))$, where $I^{'} \leq I_i, i = 1,2$.
\end{lemma}

\begin{proof}
  $I_1(a_1) \cdot I_2(a_2) \leq a_1 \cdot a_2$, so $I^{'} (I_1(a_1) \cdot I_2(a_2)) \leq I^{'} (a_1 \cdot a_2) \leq $
\end{proof}

\begin{proof}
   $\begin{array}{lll}
   & I^{'} (I_1(a_1) \cdot I_2(a_2)) \leq & \\
   & &
   \end{array}$
\end{proof}

\begin{lemma}
$ $

$I_1, \dots, I_n$ are open modalities, thus:
  $I_1(a_1) \cdot I_2(a_2) \dots \cdot I_n (a_n) \leq a$, then $I_1(a_1) \cdot \dots I_n (a_n) \leq I^{'}(a)$,
  where $I^{'} \leq I_i$ for all $i$.
\end{lemma}

\begin{proof}
\end{proof}

\begin{theorem}
  $\Gamma \rightarrow A \Rightarrow \Gamma \models A$
\end{theorem}

\end{document}
