\documentclass[a4paper]{article}
\usepackage{amsmath}
\usepackage{amsthm}
\usepackage{amsfonts}
\usepackage{amssymb}
\usepackage{bussproofs}
\usepackage{mathtools}
\usepackage{verbatim}
\usepackage{dsfont}
\usepackage{mathabx}
\usepackage[all, 2cell]{xy}
\usepackage[all]{xy}
\usepackage{wasysym}
\usepackage{rotating}
\usepackage{geometry}
\usepackage{trfsigns}
\usepackage{cmll}
\usepackage{authblk}

\newtheorem{theorem}{Theorem}
\newtheorem{prop}{Proposition}
\newtheorem{lemma}{Lemma}
\newtheorem{defin}{Definition}
\newtheorem{ex}{Example}
\newtheorem{col}{Corollary}
\newtheorem{con}{Consequence}
\usepackage{listings} 		% for source code
\date{}
\author[1,2]{Daniel Rogozin}
\affil[1]{Lomonosov Moscow State University}
\affil[2]{Serokell O\"{U}}
\title{Quantale model of Lambek calculus with subexponentials}

\begin{document}

\maketitle

\section{Calculus}

\begin{defin} A subexponential signature is an ordered quintuple:

  $\Sigma = \langle I, \preceq, W, C, E \rangle$,
\end{defin}

where $I = \{ s_1, \dots, s_n\}$, $\langle I, \preceq \rangle$ is a preorder.
$W, C, E$ are subsets of $I$ and $W \cup C \subseteq E$.

\begin{defin} Noncommutative linear logic with subexponentials ($SMALC_{\Sigma}$), where $\Sigma$ is a subexponential signature.

\begin{prooftree}
\AxiomC{$ $}
\RightLabel{\scriptsize{ax}}
\UnaryInfC{$A \rightarrow A$}
\end{prooftree}

\begin{minipage}{0.5\textwidth}
  \begin{flushleft}
    \begin{prooftree}
      \AxiomC{$\Gamma \rightarrow A$}
      \AxiomC{$\Delta, B, \Theta \rightarrow C$}
      \RightLabel{$\backslash \rightarrow$}
      \BinaryInfC{$\Delta, \Gamma, A \backslash B, \Theta \rightarrow C$}
    \end{prooftree}

    \begin{prooftree}
      \AxiomC{$\Gamma \rightarrow A$}
      \AxiomC{$\Delta, B, \Theta \rightarrow C$}
      \RightLabel{$/ \rightarrow$}
      \BinaryInfC{$\Delta, B / A, \Gamma, \Theta \rightarrow C$}
    \end{prooftree}

    \begin{prooftree}
      \AxiomC{$\Gamma, A, B, \Delta \rightarrow C$}
      \RightLabel{$\bullet \rightarrow$}
      \UnaryInfC{$\Gamma, A \bullet B, \Delta \rightarrow C$}
    \end{prooftree}

    \begin{prooftree}
      \AxiomC{$\Gamma, A_i, \Delta \rightarrow B$}
      \RightLabel{$\&, i = 1,2 \rightarrow$}
      \UnaryInfC{$\Gamma, A_1 \& A_2, \Delta \rightarrow B$}
    \end{prooftree}

    \begin{prooftree}
      \AxiomC{$\Gamma, A, \Delta \rightarrow C$}
      \AxiomC{$\Gamma, B, \Delta \rightarrow C$}
      \RightLabel{$\vee \rightarrow$}
      \BinaryInfC{$\Gamma, A \vee B, \Delta \rightarrow C$}
    \end{prooftree}

    \begin{prooftree}
      \AxiomC{$\Gamma, \Delta \rightarrow A$}
      \RightLabel{${\bf 1} \rightarrow$}
      \UnaryInfC{$\Gamma, {\bf 1}, \Delta \rightarrow A$}
    \end{prooftree}

    \begin{prooftree}
    \AxiomC{$\Gamma, A, \Delta \rightarrow C$}
    \RightLabel{$! \rightarrow$}
    \UnaryInfC{$\Gamma, !^{s} A, \Delta \rightarrow C$}
    \end{prooftree}
  \end{flushleft}
\end{minipage}
\begin{minipage}{0.5\textwidth}
  \begin{flushright}
    \begin{prooftree}
      \AxiomC{$A, \Pi \rightarrow B$}
      \RightLabel{$\rightarrow \backslash$}
      \UnaryInfC{$\Pi \rightarrow A \backslash B$}
    \end{prooftree}

    \begin{prooftree}
      \AxiomC{$\Pi, A \rightarrow B$}
      \RightLabel{$\rightarrow /$}
      \UnaryInfC{$\Pi \rightarrow B / A$}
    \end{prooftree}

    \begin{prooftree}
      \AxiomC{$\Gamma \rightarrow A$}
      \AxiomC{$\Delta \rightarrow B$}
      \RightLabel{$\rightarrow \bullet$}
      \BinaryInfC{$\Gamma, \Delta \rightarrow A \bullet B$}
    \end{prooftree}

    \begin{prooftree}
      \AxiomC{$\Gamma \rightarrow A$}
      \AxiomC{$\Gamma \rightarrow B$}
      \RightLabel{$\rightarrow \&$}
      \BinaryInfC{$\Gamma \rightarrow A \& B$}
    \end{prooftree}

    \begin{prooftree}
      \AxiomC{$\Gamma \rightarrow A_i$}
      \RightLabel{$\rightarrow \vee, i = 1,2$}
      \UnaryInfC{$\Gamma \rightarrow A_1 \vee A_2$}
    \end{prooftree}

    \begin{prooftree}
      \AxiomC{$ $}
      \RightLabel{$\rightarrow {\bf 1}$}
      \UnaryInfC{$\rightarrow {\bf 1}$}
    \end{prooftree}

    \begin{prooftree}
    \AxiomC{$!^{s_1} A_1, \dots, !^{s_n} A_n \rightarrow A$}
    \RightLabel{$\rightarrow !, \forall j, s_j \succeq s$}
    \UnaryInfC{$!^{s_1} A_1, \dots, !^{s_n} A_n \rightarrow !^{s} A$}
    \end{prooftree}
  \end{flushright}
\end{minipage}

\begin{prooftree}
  \AxiomC{$\Gamma, \Delta \rightarrow B$}
  \RightLabel{${\bf weak}_!, s \in C$}
  \UnaryInfC{$\Gamma, !^{s} A, \Delta \rightarrow B$}
\end{prooftree}

    \begin{prooftree}
    \AxiomC{$\Gamma, !^{s} A, \Delta, !^{s} A, \Theta \rightarrow B$}
    \RightLabel{${\bf ncontr}_1, s \in C$}
    \UnaryInfC{$\Gamma, !^{s} A, \Delta, \Theta \rightarrow B$}
    \end{prooftree}

    \begin{prooftree}
      \AxiomC{$\Gamma, !^{s} A, \Delta, !^{s} A, \Theta \rightarrow B$}
      \RightLabel{${\bf ncontr}_2, s \in C$}
      \UnaryInfC{$\Gamma, \Delta, !^{s} A, \Theta \rightarrow B$}
    \end{prooftree}

    \begin{prooftree}
    \AxiomC{$\Gamma, \Delta, !^{s} A, \Theta \rightarrow B$}
    \RightLabel{${\bf ex}_1, s \in E$}
    \UnaryInfC{$\Gamma, !^{s} A, \Delta, \Theta \rightarrow A$}
    \end{prooftree}

    \begin{prooftree}
      \AxiomC{$\Gamma, !^{s} A, \Delta, \Theta \rightarrow B$}
      \RightLabel{${\bf ex}_1, s \in E$}
      \UnaryInfC{$\Gamma, \Delta, !^{s} A, \Theta \rightarrow A$}
    \end{prooftree}
\end{defin}

\begin{prop}

$!_{s_i} A \leftrightarrow !_{s_i} (!_{s_i} A)$

\end{prop}

\begin{proof}
$ $

\begin{prooftree}
  \AxiomC{$A \rightarrow A$}
  \UnaryInfC{$!_{s_i} A \rightarrow A$}
  \UnaryInfC{$!_{s_i} A \rightarrow !_{s_i} A$}
  \UnaryInfC{$!_{s_i} !_{s_i} A \rightarrow !_{s_i} A$}
\end{prooftree}

\end{proof}

\section{Semantics}

\begin{defin} Quantale
$ $

  A quantale is a triple $\langle A, \bigvee, \cdot \rangle$, such that $\langle A, \bigvee \rangle$
is a complete lattice and $\langle A, \cdot \rangle$ is a semigroup. A quanlate is called unital, if $\langle A, \cdot \rangle$
is a monoid.
\end{defin}

It is easy to see, that any (unital) quantale is a residual (monoid) semigroup. We define divisions as follows:

\begin{enumerate}
\item $a \backslash b = \bigvee \{ c \: | \: a \cdot c \leq b \}$
\item $b / a = \bigvee \{ c \: | \: c \cdot a \leq b \}$
\end{enumerate}

\begin{defin}
$ $

  Let $\mathcal{Q} = \langle A, \bigvee, \cdot \rangle$ be a quantale. The center of a quantale is the set $\mathcal{Z}(\mathcal{Q}) = \{ a \in A \: | \: \forall b \in A, a \cdot b = b \cdot a \}$
\end{defin}

\begin{defin} An open modality (or quantic conucleus) on quantale $\mathcal{Q}$ is a map $\Box : \mathcal{Q} \to \mathcal{Q}$, such that

\begin{enumerate}
  \item $\Box (x) \leq x$;
  \item $\Box (x) = \Box(\Box(x))$;
  \item $x \leq y \Rightarrow \Box (x) \leq \Box (y)$;
  \item $\Box (x) \cdot \Box (y) = \Box (\Box (x) \cdot \Box (y))$.
\end{enumerate}
\end{defin}

\begin{defin}
  We define a partial order on open modalities on $\mathcal{Q}$ as $\Box_1 \leq \Box_2 \Leftrightarrow \forall a \in Q, \Box_1 (a) \leq \Box_2 (a)$.
\end{defin}

\begin{lemma}
  Let $\mathcal{Q}$ be a quantale and $\Box_{\mathcal{Q}}$ be a set of all open modalities on $\mathcal{Q}$. Then $\Box_{\mathcal{Q}}$ is a locally small category.
\end{lemma}

\begin{proof}
  $\langle \Box_{\mathcal{Q}}, \leq \rangle$ form a partial order, so $\langle \Box_{\mathcal{Q}}, \leq \rangle$ is a locally small category.
\end{proof}

\begin{lemma}
$ $

  Let $\mathcal{Q} = \langle A, \bigvee, \cdot \rangle$ be a quantale and $\Box : \mathcal{Q} \to \mathcal{Q}$ is an open modality on $\mathcal{Q}$, then
  $\Box(x) \cdot \Box(y) \leq \Box(x \cdot y)$.
\end{lemma}

\begin{proof}
$ $

  $\Box(x) \cdot \Box(y) \leq x \cdot y$, then $\Box(\Box(x) \cdot \Box(y)) \leq \Box(x \cdot y)$, but
$\Box(x) \cdot \Box(y) \leq \Box(\Box(x) \cdot \Box(y))$. Thus, $\Box(x) \cdot \Box(y) \leq \Box(x \cdot y)$.
\end{proof}

\begin{defin}
  An open modality is called central, if $\forall a, b \in Q, \Box(a) \cdot b = b \cdot \Box(a)$.
\end{defin}

\begin{defin}
  An open modality is called weak idempotent, if $\forall a, b \in Q, \Box(a) \cdot b \leq \Box(a) \cdot b \cdot \Box(a)$ and
  $b \cdot \Box(a) \leq \Box(a) \cdot b \cdot \Box(a)$.
\end{defin}

\begin{defin}
  An open modality is called unital, if $\forall a \in Q, \Box(a) \leq e$.
\end{defin}

\begin{lemma}
  Let $\Box$ be an open modality on some unital quantale $\mathcal{Q} = \langle A, \bigvee, \cdot, e \rangle$.
  Then, if $\Box$ is unital and weak idempotent, then $\Box$ is central.
\end{lemma}

\begin{proof}
$ $

  $\begin{array}{lll}
  & b \cdot \Box(a) \leq & \\
  & \:\:\:\: \text{Right weak idempotence}& \\
  &\Box(a) \cdot b \cdot \Box(a) \leq & \\
  & \:\:\:\: \text{Unitality}& \\
  & \Box(a) \cdot b \cdot e \leq & \\
  & \:\:\:\: \text{Identity}& \\
  &\Box(a) \cdot b \leq & \\
  & \:\:\:\: \text{Left weak idempotence}& \\
  &\Box(a) \cdot b \cdot \Box(a) \leq & \\
  & \:\:\:\: \text{Unitality}& \\
  &e \cdot b \cdot \Box(a) \leq & \\
  & \:\:\:\: \text{Identity}& \\
  &b \cdot \Box(a)&
  \end{array}$

Hence, $b \cdot \Box(a) = \Box(a) \cdot b$, so $\forall a \in A, \Box(a) \in \mathcal{Z}(\mathcal{Q})$.

\end{proof}

\begin{prop}
$ $

  Let $\mathcal{Q}$ be a quantale and $S \subseteq \mathcal{Q}$ a subquantale, then $\Box : \mathcal{Q} \to \mathcal{Q}$, such that
$\Box(a) = \bigvee \{ s \in S \: | \: x \leq a \}$, is an open modality.
\end{prop}

\begin{proof}
  See
\end{proof}

\begin{prop}
$ $

  Let $\mathcal{Q}$ be a quantale and $S_1 \subseteq S_2 \subseteq \mathcal{Q}$.

  Then $\Box_1 (a) \leq \Box_2 (a)$.
\end{prop}

\begin{proof}
$ $

  Let $a \in Q$, so $\{ s \in S_1 \: | \: s \leq a \} \subseteq \{ s \in S_2 \: | \: s \leq a \}$, so
  $\bigvee \{ s \in S_1 \: | \: s \leq a \} \subseteq \bigvee \{ s \in S_2 \: | \: s \leq a \}$.
  Thus, $\Box_1 (a) \leq \Box_2 (a)$.
\end{proof}

\begin{prop}
$ $

Let $\mathcal{Q}$ be a quantale and $S \subseteq \mathcal{Q}$ a subquantale, then the following operations are open modalities:

\begin{enumerate}
  \item $\Box_z (a) = \bigvee \{ s \in S \: | s \leq a, s \in \mathcal{Z}(\mathcal{Q}) \}$;
  \item $\Box_{\mathds{1}} (a) = \bigvee \{ s \in S \: | s \leq a, s \leq \mathds{1} \}$;
  \item $\Box_{idem} (a) = \bigvee \{ s \in S \: | s \leq a, \forall b \in Q, b \cdot s \vee s \cdot b \leq s \cdot b \cdot s\}$;
  \item $\Box_{z, \mathds{1}}, I_{z, idem}, I_{\mathds{1}, idem}, I_{z, \mathds{1}, idem}$.
\end{enumerate}
\end{prop}

\begin{proof}
  Immediatly.
\end{proof}

\begin{prop}
$ $

\begin{enumerate}
  \item $\forall a \in \mathcal{Q}, \Box_{\mathds{1}, idem}(a) \leq \Box_z (a)$.
  \item $\forall a \in \mathcal{Q}, \Box_{z, \mathds{1}, idem} = \Box_{\mathds{1}, idem}(a)$
\end{enumerate}

\end{prop}

\begin{proof}
  Follows from Lemma 3.
\end{proof}

\begin{prop}
$ $

\begin{enumerate}
  \item $\Box_z (a) \vee \Box_{\mathds{1}} (a) \vee \Box_{idem} (a) \leq \Box(a)$
  \item $\Box_{z, \mathds{1}, idem} \leq \Box_{z, \mathds{1}} (a) \wedge \Box_{z, idem} (a)$
\end{enumerate}
\end{prop}

\begin{lemma}
  $\forall a \in Q, \Box_1 (a) \leq \Box_2 (\Box_1 (a))$, if $\Box_1 (a) \leq \Box_2 (a)$.
\end{lemma}

\begin{proof}
  $\Box_1 (a) \leq \Box_1 (\Box_1 (a)) \leq \Box_2 (\Box_1 (a))$
\end{proof}

\begin{lemma}
  $\Box_1(a_1) \cdot \Box_2(a_2) \leq \Box^{'} (\Box_1(a_1) \cdot \Box_2(a_2))$, where $\Box_i \leq \Box^{'}, i = 1,2$.
\end{lemma}

\begin{proof}
$ $

  $\begin{array}{lll}
  &\Box_1(a_1) \cdot \Box_2(a_2) \leq & \\
  &\Box_1 (\Box_1 (a_1)) \cdot \Box_2 (\Box_2 (a_2)) \leq & \\
  &\Box^{'} (\Box_1 (a_1)) \cdot \Box^{'} (\Box_2 (a_2)) \leq & \\
  &\Box^{'}(\Box_1 (a_1) \cdot \Box_2 (a_2))&
  \end{array}$
\end{proof}

\begin{defin} Interpretation of subexponential signature

  Let $\Sigma = \langle I, \preceq, W, C, E \rangle$ be a subexponential signature, where $|I| = n$ and
  $\Box_{\mathcal{Q}}$ is a category of open modalities on a quantale $\mathcal{Q}$.
  Subexponential interpretation is a contravariant functor $\sigma : I \to \Box_{\mathcal{Q}}$ defined as follows:

  $\sigma(s_i) = \begin{cases}
  \Box_i : \mathcal{Q} \to \mathcal{Q} \text{, s.t.} \forall a \in Q, \Box_i(a) = \{ s \in S_i \: | \: s \leq a\},
  \\ \:\:\:\: \text{if $s_i \notin W \cap C \cap E$} \\
  \Box_i : \mathcal{Q} \to \mathcal{Q} \text{, s.t.} \forall a \in Q, \Box_i(a) = \{ s \in S_i \: | \: s \leq a, s \leq \mathds{1}\},
  \\ \:\:\:\: \text{if $s_i \in W$} \\
  \Box_i : \mathcal{Q} \to \mathcal{Q} \text{, s.t.} \forall a \in Q, \Box_i(a) = \{ s \in S_i \: | \: s \leq a, s \in \mathcal{Z}(\mathcal{Q}) \},
  \\ \:\:\:\: \text{if $s_i \in E$} \\
  \Box_i : \mathcal{Q} \to \mathcal{Q} \text{, s.t.} \forall a \in Q, \Box_i(a) = \{ s \in S_i \: | \: s \leq a, \forall b, b \cdot s \vee s \cdot b \leq s \cdot b \cdot s \},
  \\ \:\:\:\: \text{if $s_i \in E$} \\
  \text{otherwise, if $s_i$ belongs to some intersection of subsets, then we combine the relevant conditions } \\
  \end{cases}$
\end{defin}

\begin{defin} Let $\mathcal{Q}$ be a quantale, $f : Tp \to \mathcal{Q}$ a valuation and $\sigma : I \to \Box_{\mathcal{Q}}$ a subexponential interpretation, then
  interpretation is defined inductively:

\begin{center}
$\begin{array}{lll}
& [\![p_i]\!] = f(p_i)&\\
& [\![\mathds{1}]\!] = e & \\
&[\![A \bullet B]\!] = [\![A]\!] \cdot [\![B]\!] & \\
&[\![A \backslash B]\!] = [\![A]\!] \backslash [\![B]\!] & \\
&[\![A / B]\!] = [\![A]\!] / [\![B]\!]& \\
&[\![A \& B]\!] = [\![A]\!] \wedge [\![B]\!]& \\
&[\![A \vee B]\!] = [\![A]\!] \vee [\![B]\!]& \\
&[\![!_{s_i} A]\!] = \sigma(s_i) [\![A]\!]&
\end{array}$
\end{center}
\end{defin}

\begin{defin}
  $\Gamma \models A \Leftrightarrow \forall f, \forall \sigma, [\![\Gamma]\!] \leq [\![A]\!]$
\end{defin}

\begin{theorem}
  $\Gamma \rightarrow A \Rightarrow [\![\Gamma]\!] \leq [\![A]\!]$
\end{theorem}

\begin{proof}
We consider cases with modal rules.

\begin{enumerate}

\item Let $!_{s_1} A_1, \dots, !_{s_n} A_n \rightarrow A$ and $\forall i, s \preceq s_i$.

Then $\forall a \in Q, \sigma(s_i)(a) \leq \sigma(s)(a)$.

By IH, $\sigma(s_1)[\![A_1]\!] \cdot \dots \cdot \sigma(s_n) [\![A_n]\!] \leq [\![A]\!]$.

Thus, $\sigma(s)(\sigma(s_1)[\![A_1]\!] \cdot \dots \cdot \sigma(s_n) [\![A_n]\!]) \leq \sigma(s)([\![A]\!])$.

By Lemma 5, $\sigma(s_1)[\![A_1]\!] \cdot \dots \cdot \sigma(s_n) [\![A_n]\!] \leq \sigma(s)(\sigma(s_1)[\![A_1]\!] \cdot \dots \cdot \sigma(s_n) [\![A_n]\!])$.

So, $\sigma(s_1)[\![A_1]\!] \cdot \dots \cdot \sigma(s_n) [\![A_n]\!] \leq \sigma(s)([\![A]\!])$.

\item Let $\Gamma, A, \Delta \rightarrow B$.

By IH, $[\![\Gamma]\!] \cdot [\![A]\!] \cdot [\![\Delta]\!] \leq [\![B]\!]$.

By the definition, $\sigma(s_i)([\![A]\!]) \leq [\![A]\!]$.

So, $[\![\Gamma]\!] \cdot \sigma(s_i)([\![A]\!]) \cdot [\![\Delta]\!] \leq [\![B]\!]$

\item Let $\Gamma, \Delta \rightarrow B$, $A \in Fm$, and $s_i \in W$.

So, $[\![\Gamma]\!] \cdot [\![\Delta]\!] \leq [\![B]\!]$,
then $[\![\Gamma]\!] \cdot e \cdot [\![\Delta]\!] \leq [\![B]\!]$, where $e \in Q$ is unit.

By the definition of unital open modality, $\sigma(s_i)([\![A]\!]) \leq e$.

Thus, $[\![\Gamma]\!] \cdot \sigma(s_i)([\![A]\!]) \cdot [\![\Delta]\!] \leq [\![B]\!]$.

\item Let $\Gamma, !_{s_i} A, \Delta, !_{s_i} A, \Pi \rightarrow B$ and $s_i \in C$.

By IH, $[\![\Gamma]\!] \cdot \sigma(s_i)([\![A]\!]) \cdot [\![\Delta]\!] \cdot \sigma(s_i)([\![A]\!]) \cdot [\![\Pi]\!] \leq [\![B]\!]$.

By the definition, $\sigma(s_i)([\![A]\!]) \cdot [\![\Delta]\!] \leq \sigma(s_i)([\![A]\!]) \cdot [\![\Delta]\!] \cdot \sigma(s_i)([\![A]\!])$.

Then $[\![\Gamma]\!] \cdot \sigma(s_i)([\![A]\!]) \cdot [\![\Delta]\!] \cdot [\![\Pi]\!] \leq [\![B]\!]$

\item Let $\Gamma, !_{s_i} A, \Delta, \Pi \rightarrow B$ and $s_i \in E$, so $\sigma(s_i)(a) \in \mathcal{Z}(\mathcal{Q})$ for all $a \in Q$ by the definition.

By IH, $[\![\Gamma]\!] \cdot \sigma(s_i)([\![A]\!]) \cdot [\![\Delta]\!] \cdot [\![\Pi]\!] \leq [\![B]\!]$

Hence, $[\![\Gamma]\!] \cdot [\![\Delta]\!] \cdot \sigma(s_i)([\![A]\!]) \cdot [\![\Pi]\!] \leq [\![B]\!]$.

\end{enumerate}

\end{proof}

\section{Quantale completeness}

\begin{defin}
$ $

  Let $\mathcal{F} \subseteq Fm$, an ideal is a subset $\mathcal{I} \subseteq \mathcal{F}$, such that:

\begin{itemize}
  \item If $B \in \mathcal{I}$ and $A \rightarrow B$, then $A \in \mathcal{I}$;
  \item If $A, B \in \mathcal{I}$, then $A \lor B \in \mathcal{I}$.
\end{itemize}
\end{defin}

\begin{defin}
$ $

  Let $S \subseteq \mathcal{F} \subseteq Fm$,
  then $\bigvee S = \bigcap \{ \mathcal{I} \subseteq \mathcal{F} \: | \: S \subseteq \mathcal{I} \}$
\end{defin}

\begin{prop}
  $\bigvee S$ is an ideal.
\end{prop}

\begin{lemma}
  $A \subseteq Fm$, then $\{ B \: | \: B \rightarrow A' \} = \bigvee A$.
\end{lemma}

\begin{proof}
$ $

Let $A \subseteq Fm$. Then $\{ B \: | \: B \rightarrow A', A' \in A \} \subseteq \bigvee A$, so far as $\bigvee A$ is an ideal.

On the other hand, $\{ B \: | \: B \rightarrow A', A' \in A \}$ is an ideal, it is easy to see that this set is closed under $\lor$.
So, $\bigvee A \subseteq \{ B \: | \: B \rightarrow A', A' \in A \}$.
\end{proof}

\begin{lemma}
  $\bigvee A \subseteq \bigvee B$ iff $\forall A' \in A, \forall B' \in B, A' \rightarrow B'$.
\end{lemma}

\begin{proof}
  Let $\bigvee A \subseteq \bigvee B$,
  then $\{ C | C \rightarrow A', A' \in A \} \subseteq \{ D \: | \: D \rightarrow B', B' \in B \}$.

Thus, for all $A' \in A$, $A' \in \{ C | C \rightarrow A', A' \in A \}$,
then $A' \in \{ D \: | \: D \rightarrow B', B' \in B \}$, hence $A' \rightarrow B'$, for all $B' \in B$.

On the other hand, let $A' \rightarrow B'$ for all $A' \in A$, $B' \in B$ and $C \in \bigvee A$.

Thus, $C \rightarrow A'$, then $C \rightarrow B'$ by cut, so $C \in B'$.

\end{proof}

\begin{lemma}
  Let $\mathcal{Q} = \{ \bigvee S \: | \: S \subseteq Fm \}$ and $\bigvee \mathcal{A} \cdot \bigvee \mathcal{B} =
  \bigvee \{ A \bullet B \: | \: A \in \mathcal{A}, B \in \mathcal{B} \}$.
  Then $\langle \mathcal{Q}, \subseteq, \cdot, \bigvee{{\bf 1}}\rangle$ is a quantale.
\end{lemma}

\begin{proof}
  See
\end{proof}

\begin{lemma}
  Let $!_s \in I$, then $\Box_s (\bigvee A) = \bigvee \{ B \: | \: B \rightarrow !_s A^{'}, A^{'} \in A \}$
  is a quantic conucleus.
\end{lemma}

\begin{proof}
$ $

\begin{enumerate}
  \item $\Box_s (\bigvee A) \subseteq \bigvee A$;

Let $B \in \Box_s (\bigvee A)$, then for all $A^{'} \in A$, $B \rightarrow !_s A^{'}$,
but $!_s A^{'} \rightarrow A^{'}$, then $B \rightarrow A^{'}$, so $B \in \bigvee A$.

\item $\Box_s (\Box_s(\bigvee A)) = \bigvee \Box_s (\bigvee A)$;

$\begin{array}{lll}
& \Box_s (\Box_s(\bigvee A)) = & \\
& \{ B \: | \: B \rightarrow !_s !_s A^{'}, A^{'} \in A \} = & \\
& \{ B \: | \: B \rightarrow !_s A^{'}, A^{'} \in A \} &
\end{array}$,
that follows from equivalence $!_s !_s B \leftrightarrow !_s B$.

\item $\bigvee A \subseteq \bigvee B \Rightarrow \Box_s (\bigvee A) \subseteq \Box_s (\bigvee B)$;

Follows from admissiability of K-rule for all $s \in I$.

\item $\Box_s \bigvee A \cdot \Box_s \bigvee B = \Box_s (\Box_s \bigvee A \cdot \Box_s \bigvee B)$.

$\begin{array}{lll}
&\Box_s \bigvee A \cdot \Box_s \bigvee B = & \\
&\bigvee \{ C \bullet D \: | \: C \bullet D \rightarrow !_s A^{'} \bullet !_s B^{'} \} = & \\
&\bigvee \{ C \bullet D \: | \: C \bullet D \rightarrow !_s (!_s A^{'} \bullet !_s B^{'})\} = & \\
&\Box_s (\Box_s \bigvee A \cdot \Box_s \bigvee B)&
\end{array}$

\end{enumerate}
\end{proof}

\begin{lemma}
$ $

  \begin{enumerate}
    \item Let $s \in W$, then for all $A \subseteq Fm$, ${\bf 1} \in \Box_s (\bigvee A)$;
    \item Let $s \in E$, then $\Box_s (\bigvee A) \cdot \bigvee B = \bigvee B \cdot \Box_s (\bigvee A)$.
    \item Let $s \in C$, then $(\Box_s \bigvee A \cdot \bigvee B) \cup (\bigvee B \cdot \Box_s \bigvee A) \subseteq \Box_s \bigvee A \cdot \bigvee B \cdot \Box_s \bigvee A$, for all $B \subseteq Fm$.
  \end{enumerate}
\end{lemma}

\begin{proof}
  \begin{enumerate}
    \item Let $s \in W$, then for all $A \subseteq Fm$, $\Box_s (\bigvee A) = \{ !_s B \: | \: !_s B \rightarrow A^{'}, A^{'} \in A \}$.
    But, $!_s B \rightarrow {\bf 1}$, hence, $1 \in \Box_s (\bigvee A)$, so far as $\Box_s (\bigvee A)$ is an ideal.
    \item  $ $

    $\begin{array}{lll}
    &\Box_s (\bigvee A) \cdot \bigvee B = & \\
    &\bigvee \{ !_s C \bullet D \: | \: !_s C \bullet D \rightarrow A^{'} \bullet B^{'}, A^{'} \in A, B^{'} \in B \} = & \\
    &\bigvee \{ D \bullet !_s C \: | \: D \bullet !_s C \rightarrow A^{'} \bullet B^{'}, A^{'} \in A, B^{'} \in B \} = & \\
    &\bigvee B \cdot \Box_s (\bigvee A)&
    \end{array}$

    \item $ $

    $\Box_s \bigvee A \cdot \bigvee B = \bigvee \{ !_s C \bullet D \: | \: !_s C \bullet D \rightarrow A^{'} \bullet B^{'}\}$.
    $!_s C \bullet D \rightarrow !_s C \bullet D \bullet !_s C$, hence $\Box_s \bigvee A \cdot \bigvee B \subseteq \Box_s \bigvee A \cdot \bigvee B \cdot \Box_s \bigvee A$.

    Similarly with $\bigvee B \cdot \Box_s \bigvee A$.
  \end{enumerate}
\end{proof}

\begin{lemma}
$ $

  Let $i, j \in I$ and $i \preceq j$, then for all $A \subseteq Fm$, $\Box_j (\bigvee A) \subseteq \Box_i (\bigvee A)$.
\end{lemma}

\begin{proof}
  \begin{enumerate}

    Let $i, j \in I$ and $i \preceq j$. Let $B \in \Box_j (\bigvee A)$, then $\forall A^{'}$, $B \rightarrow !_j A^{'}$.

    But $!_j A \rightarrow !_i A$. Then $B \rightarrow !_i A$ by hence. So, $B \in \Box_j (\bigvee A)$.
  \end{enumerate}
\end{proof}

\begin{defin} Let $Q$ be a syntactic quantale as proposed above and
  $\mathcal{I} = \langle I, \preceq, W, C, E \rangle$ be a subexponential signature.

  We define a map $\Box : \mathcal{I} \to Mod_{\mathcal{Q}}$ as follows:

  $\Box(i)(\bigvee A) = \{ B \: | \: B \rightarrow !_i A \}$.
\end{defin}

\begin{lemma} $\Box$ is a subexponential interpretation.
\end{lemma}

\begin{proof}
  Follows from lemmas 10 and 11.
\end{proof}

\begin{lemma}
$ $

  Let $Q$ be a quantale constructed above and $\Box_1, \dots, \Box_n$ be a family of quantic conuclei on $Q$.
  Then there exist a model $\langle Q, [\![.]\!]\rangle$, such that $[\![A]\!] = \bigvee \{ A \}$, $A \in Fm$.
\end{lemma}

\begin{proof}
$ $

  We define an interpreation as follows:

\begin{enumerate}
  \item $[\![p_i]\!] = \bigvee \{ p_i \}$
  \item $[\![{\bf 1}]\!] = \bigvee \{ {\bf 1} \}$
  \item $[\![A \bullet B]\!] = \bigvee \{ A \bullet B \}$
  \item $[\![A / B]\!] = \bigvee \{ A / B \}$
  \item $[\![B \setminus A]\!] = \bigvee \{ B \setminus A \}$
  \item $[\![A \& B ]\!] = \bigvee \{ A \& B \}$
  \item $[\![A \lor B]\!] = \bigvee \{ A \lor B\}$
  \item $[\![!_s A]\!] = \Box(s) (\bigvee A) = \{ B \: | \: B \rightarrow !_s A \} = \bigvee \{ !_s A \}$
\end{enumerate}
\end{proof}

\begin{theorem}
  $\Gamma \models A \Rightarrow \Gamma \rightarrow A$.
\end{theorem}

\begin{proof}
  Follows from lemmas 6, 12, 13.
\end{proof}

\end{document}
