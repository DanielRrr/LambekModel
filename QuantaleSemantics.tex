\documentclass[a4paper]{article}
\usepackage{amsmath}
\usepackage{amsthm}
\usepackage{amsfonts}
\usepackage{amssymb}
\usepackage{bussproofs}
\usepackage{mathtools}
\usepackage{verbatim}
\usepackage{dsfont}
\usepackage{mathabx}
\usepackage[all, 2cell]{xy}
\usepackage[all]{xy}
\usepackage{wasysym}
\usepackage{rotating}
\usepackage{geometry}
\usepackage{trfsigns}
\usepackage{cmll}
\newtheorem{theorem}{Theorem}
\newtheorem{prop}{Proposition}
\newtheorem{lemma}{Lemma}
\newtheorem{defin}{Definition}
\newtheorem{ex}{Example}
\newtheorem{col}{Corollary}
\newtheorem{con}{Consequence}
\usepackage{listings} 		% for source code
\date{}
\title{Quantale model of noncommutative linear logic with subexponentials}
\begin{document}

\maketitle

\section{Calculus}

\begin{defin} A subexponential signature is an ordered quintuple:

  $\Sigma = \langle I, \preceq, W, C, E \rangle$,
\end{defin}

where $I = \{ s_1, \dots, s_n\}$, $\langle I, \preceq \rangle$ is a preorder.
$W, C, E$ are subsets of $I$ and $W \cup C \subseteq E$.

\begin{defin} Noncommutative linear logic with subexponentials ($SMALC_{\Sigma}$), where $\Sigma$ is a subexponential signature.

\begin{prooftree}
\AxiomC{$ $}
\RightLabel{\scriptsize{ax}}
\UnaryInfC{$A \Rightarrow A$}
\end{prooftree}

\begin{minipage}{0.5\textwidth}
  \begin{flushleft}
    \begin{prooftree}
      \AxiomC{$\Gamma \Rightarrow A$}
      \AxiomC{$\Delta, B, \Theta \Rightarrow C$}
      \RightLabel{$\backslash \rightarrow$}
      \BinaryInfC{$\Delta, \Gamma, A \backslash B, \Theta \Rightarrow C$}
    \end{prooftree}

    \begin{prooftree}
      \AxiomC{$\Gamma \Rightarrow A$}
      \AxiomC{$\Delta, B, \Theta \Rightarrow C$}
      \RightLabel{$/ \rightarrow$}
      \BinaryInfC{$\Delta, B / A, \Gamma, \Theta \Rightarrow C$}
    \end{prooftree}

    \begin{prooftree}
      \AxiomC{$\Gamma, A, B, \Delta \Rightarrow C$}
      \RightLabel{$\bullet \rightarrow$}
      \UnaryInfC{$\Gamma, A \bullet B, \Delta \Rightarrow C$}
    \end{prooftree}

    \begin{prooftree}
      \AxiomC{$\Gamma, A_i, \Delta \Rightarrow B$}
      \RightLabel{$\&, i = 1,2 \rightarrow$}
      \UnaryInfC{$\Gamma, A_1 \& A_2, \Delta \Rightarrow B$}
    \end{prooftree}

    \begin{prooftree}
      \AxiomC{$\Gamma, A, \Delta \Rightarrow C$}
      \AxiomC{$\Gamma, B, \Delta \Rightarrow C$}
      \RightLabel{$\vee \rightarrow$}
      \BinaryInfC{$\Gamma, A \vee B, \Delta \Rightarrow C$}
    \end{prooftree}

    \begin{prooftree}
      \AxiomC{$\Gamma, \Delta \Rightarrow A$}
      \RightLabel{${\bf 1} \rightarrow$}
      \UnaryInfC{$\Gamma, {\bf 1}, \Delta \Rightarrow A$}
    \end{prooftree}

    \begin{prooftree}
    \AxiomC{$\Gamma, A, \Delta \Rightarrow C$}
    \RightLabel{$! \rightarrow$}
    \UnaryInfC{$\Gamma, !^{s} A, \Delta \Rightarrow C$}
    \end{prooftree}
  \end{flushleft}
\end{minipage}
\begin{minipage}{0.5\textwidth}
  \begin{flushright}
    \begin{prooftree}
      \AxiomC{$A, \Pi \Rightarrow B$}
      \RightLabel{$\rightarrow \backslash$}
      \UnaryInfC{$\Pi \Rightarrow A \backslash B$}
    \end{prooftree}

    \begin{prooftree}
      \AxiomC{$\Pi, A \Rightarrow B$}
      \RightLabel{$\rightarrow /$}
      \UnaryInfC{$\Pi \Rightarrow B / A$}
    \end{prooftree}

    \begin{prooftree}
      \AxiomC{$\Gamma \Rightarrow A$}
      \AxiomC{$\Delta \Rightarrow B$}
      \RightLabel{$\rightarrow \bullet$}
      \BinaryInfC{$\Gamma, \Delta \Rightarrow A \bullet B$}
    \end{prooftree}

    \begin{prooftree}
      \AxiomC{$\Gamma \Rightarrow A$}
      \AxiomC{$\Gamma \Rightarrow B$}
      \RightLabel{$\rightarrow \&$}
      \BinaryInfC{$\Gamma \Rightarrow A \& B$}
    \end{prooftree}

    \begin{prooftree}
      \AxiomC{$\Gamma \Rightarrow A_i$}
      \RightLabel{$\rightarrow \vee, i = 1,2$}
      \UnaryInfC{$\Gamma \Rightarrow A_1 \vee A_2$}
    \end{prooftree}

    \begin{prooftree}
      \AxiomC{$ $}
      \RightLabel{$\rightarrow {\bf 1}$}
      \UnaryInfC{$\Rightarrow {\bf 1}$}
    \end{prooftree}

    \begin{prooftree}
    \AxiomC{$!^{s_1} A_1, \dots, !^{s_n} A_n \Rightarrow A$}
    \RightLabel{$\rightarrow !, \forall j, s_j \succeq s$}
    \UnaryInfC{$!^{s_1} A_1, \dots, !^{s_n} A_n \Rightarrow !^{s} A$}
    \end{prooftree}
  \end{flushright}
\end{minipage}

\begin{prooftree}
  \AxiomC{$\Gamma, \Delta \Rightarrow B$}
  \RightLabel{${\bf weak}_!, s \in C$}
  \UnaryInfC{$\Gamma, !^{s} A, \Delta \Rightarrow B$}
\end{prooftree}

    \begin{prooftree}
    \AxiomC{$\Gamma, !^{s} A, \Delta, !^{s} A, \Theta \Rightarrow B$}
    \RightLabel{${\bf ncontr}_1, s \in C$}
    \UnaryInfC{$\Gamma, !^{s} A, \Delta, \Theta \Rightarrow B$}
    \end{prooftree}

    \begin{prooftree}
      \AxiomC{$\Gamma, !^{s} A, \Delta, !^{s} A, \Theta \Rightarrow B$}
      \RightLabel{${\bf ncontr}_2, s \in C$}
      \UnaryInfC{$\Gamma, \Delta, !^{s} A, \Theta \Rightarrow B$}
    \end{prooftree}

    \begin{prooftree}
    \AxiomC{$\Gamma, \Delta, !^{s} A, \Theta \Rightarrow B$}
    \RightLabel{${\bf ex}_1, s \in E$}
    \UnaryInfC{$\Gamma, !^{s} A, \Delta, \Theta \Rightarrow A$}
    \end{prooftree}

    \begin{prooftree}
      \AxiomC{$\Gamma, !^{s} A, \Delta, \Theta \Rightarrow B$}
      \RightLabel{${\bf ex}_1, s \in E$}
      \UnaryInfC{$\Gamma, \Delta, !^{s} A, \Theta \Rightarrow A$}
    \end{prooftree}
\end{defin}

\begin{lemma}
  Let $A \Leftrightarrow B$, then $C [p_i := A] \Leftrightarrow C [p_i := B]$
\end{lemma}

\begin{proof}
  By induction on $C$.
\end{proof}

\begin{lemma}
  \begin{itemize}
  \item $!_{s_i} \Gamma \rightarrow A$ iff $!_{s_i} \Gamma \rightarrow !_{s_i} A$.
  \item $!_{s_i} A \leftrightarrow !_{s_i} (!_{s_i} A)$
\end{itemize}
\end{lemma}

\begin{proof}
$ $

\begin{enumerate}
\item $!_{s_i} \Gamma \rightarrow A$ iff $!_{s_i} \Gamma \rightarrow !_{s_i} A$;
  \begin{prooftree}
    \AxiomC{$!_{s_i} \Gamma  \rightarrow A$}
    \RightLabel{$\to !_{s_i}$}
    \UnaryInfC{$!_{s_i} \Gamma  \rightarrow !_{s_i} A$}
  \end{prooftree}

\begin{prooftree}
  \AxiomC{$!_{s_i} \Gamma \rightarrow !_{s_i} A$}
  \AxiomC{$A \rightarrow A$}
  \RightLabel{$!_{s_i} \to$}
  \UnaryInfC{$!_{s_i} A \rightarrow A$}
  \RightLabel{\text{cut}}
  \BinaryInfC{$!_{s_i} \Gamma \rightarrow A$}
\end{prooftree}

\item $!_{s_i} A \leftrightarrow !_{s_i} !_{s_i} A$

\begin{prooftree}
  \AxiomC{$A \rightarrow A$}
  \UnaryInfC{$!_{s_i} A \rightarrow A$}
  \UnaryInfC{$!_{s_i} A \rightarrow !_{s_i} A$}
  \UnaryInfC{$!_{s_i} !_{s_i} A \rightarrow !_{s_i} A$}
\end{prooftree}
\end{enumerate}
\end{proof}

\section{Semantics}

\begin{defin} Quantale
$ $

  A quantale is a triple $\langle A, \bigvee, \cdot \rangle$, such that $\langle A, \bigvee \rangle$
is a complete lattice and $\langle A, \cdot \rangle$ is a semigroup. A quanlate is called unital, if $\langle A, \cdot \rangle$
is a monoid.
\end{defin}

It is easy to see, that any (unital) quantale is a residual (monoid) semigroup. We define divisions as follows:

\begin{enumerate}
\item $a \backslash b = \bigvee \{ c \: | \: a \cdot c \leq b \}$
\item $b / a = \bigvee \{ c \: | \: c \cdot a \leq b \}$
\end{enumerate}

\begin{defin}
$ $

  Let $\langle A, \bigvee, \cdot \rangle$ be a quantale. The center of a quantale is the set $Z(Q) = \{ a \in Q \: | \: \forall b \in Q, a \cdot b = b \cdot a \}$
\end{defin}

\begin{defin} An open modality (or quantic conucleus) on quantale $Q$ is a map $I : Q \to Q$, such that

\begin{enumerate}
  \item $I(x) \leq x$;
  \item $I(x) = I(I(x))$;
  \item $x \leq y \Rightarrow I(x) \leq I(y)$;
  \item $I(x) \cdot I(y) = I(I(x) \cdot I(y))$.
\end{enumerate}
\end{defin}

\begin{lemma}
$ $

  Let $\langle A, \bigvee, \cdot \rangle$ be a quantale and $I : Q \to Q$ is an open modality on $Q$, then
  $I(x) \cdot I(y) \leq I(x \cdot y)$.
\end{lemma}

\begin{proof}
$ $

  $I(x) \cdot I(y) \leq x \cdot y$, then $I(I(x) \cdot I(y)) \leq I(x \cdot y)$, but
$I(x) \cdot I(y) \leq I(I(x) \cdot I(y))$. Thus, $I(x) \cdot I(y) \leq I(x \cdot y)$.
\end{proof}

\begin{defin}
  An open modality is called central, if $\forall a, b \in Q, I(a) \cdot b = b \cdot I(a)$.
\end{defin}

\begin{defin}
  An open modality is called weak idempotent, if $\forall a, b \in Q, I(a) \cdot b \leq I(a) \cdot b \cdot I(a)$ and
  $b \cdot I(a) \leq I(a) \cdot b \cdot I(a)$.
\end{defin}

\begin{defin}
  An open modality is called unital, if $\forall a \in Q, I(a) \leq e$.
\end{defin}

\begin{lemma}
  Let $I$ be an interior on some unital quantale $\langle Q, \bigvee, \cdot, e \rangle$.
  Then, if $I$ is unital and weak idempotent, then $I$ is central.
\end{lemma}

\begin{proof}
$ $

  $\begin{array}{lll}
  & b \cdot I(a) \leq & \\
  & \:\:\:\: \text{Right weak idempotence}& \\
  &I(a) \cdot b \cdot I(a) \leq & \\
  & \:\:\:\: \text{Unitality}& \\
  & I(a) \cdot b \cdot I(e) \leq & \\
  & \:\:\:\: \text{Identity}& \\
  &I(a) \cdot b \leq & \\
  & \:\:\:\: \text{Left weak idempotence}& \\
  &I(a) \cdot b \cdot I(a) \leq & \\
  & \:\:\:\: \text{Unitality}& \\
  &e \cdot b \cdot I(a) \leq & \\
  & \:\:\:\: \text{Identity}& \\
  &b \cdot I(a)&
  \end{array}$

Hence, $b \cdot I(a) = I(a) \cdot b$

\end{proof}

\begin{prop}
$ $

  Let $Q$ be a quantale and $S \subseteq Q$ a subquantale, then $I : Q \to Q$, such that
$I(a) = \bigvee \{ s \in S \: | \: x \leq a \}$, is an open modality. Moreover, $\{ x \in Q \: | \: I(x) = x\} = S$.
\end{prop}

\begin{proof}
  See
\end{proof}

\begin{prop}
$ $

  Let $Q$ be a quantale and $S_1, S_2 \subseteq Q$, such that $S_1 \subseteq S_2$.

  Then $I_1 (a) \leq I_2 (a)$.
\end{prop}

\begin{proof}
$ $

  Let $a \in Q$, so $\{ s \in S_1 \: | \: s \leq a \} \subseteq \{ s \in S_2 \: | \: s \leq a \}$, so
  $\bigvee \{ s \in S_1 \: | \: s \leq a \} \subseteq \bigvee \{ s \in S_2 \: | \: s \leq a \}$.
  Thus, $I_1 (a) \leq I_2 (a)$.
\end{proof}

\begin{prop}
$ $

Let $Q$ be a quantale and $S \subseteq Q$ a subquantale, then the following operations are open modalities:

\begin{enumerate}
  \item $I_z (a) = \bigvee \{ s \in S \: | s \leq a, s \in Z(Q) \}$;
  \item $I_{\mathds{1}} (a) = \bigvee \{ s \in S \: | s \leq a, s \leq \mathds{1} \}$;
  \item $I_{idem} (a) = \bigvee \{ s \in S \: | s \leq a, \forall b \in Q, b \cdot s \vee s \cdot b \leq s \cdot b \cdot s\}$;
  \item $I_{z, \mathds{1}}, I_{z, idem}, I_{\mathds{1}, idem}, I_{z, \mathds{1}, idem}$.
\end{enumerate}
\end{prop}

\begin{proof}
  Immediatly.
\end{proof}

\begin{prop}
$ $

\begin{enumerate}
  \item $\forall a \in Q, I_{\mathds{1}, idem}(a) \leq I_z (a)$.
  \item $\forall a \in Q, I_{z, \mathds{1}, idem} = I_{\mathds{1}, idem}(a)$
\end{enumerate}

\end{prop}

\begin{proof}
  Follows from Lemma 3.
\end{proof}

\begin{prop}
$ $

\begin{enumerate}
  \item $I_z (a) \vee I_{\mathds{1}} (a) \vee I_{idem} (a) \leq I(a)$
  \item $I_{z, \mathds{1}, idem} \leq I_{z, \mathds{1}} (a) \wedge I_{z, idem} (a)$
\end{enumerate}
\end{prop}

\begin{lemma}
  $\forall a \in Q, I_1 (a) \leq I_2 (I_1 (a))$, if $I_1 (a) \leq I_2 (a)$.
\end{lemma}

\begin{proof}
  $I_1 (a) \leq I_1 (I_1 (a)) \leq I_2 (I_1 (a))$
\end{proof}

\begin{lemma}
  $I_1(a_1) \cdot I_2(a_2) \leq I^{'} (I_1(a_1) \cdot I_2(a_2))$, where $I_i \leq I^{'}, i = 1,2$.
\end{lemma}

\begin{proof}
$ $

  $\begin{array}{lll}
  &I_1(a_1) \cdot I_2(a_2) \leq & \\
  &I_1 (I_1 (a_1)) \cdot I_2 (I_2 (a_2)) \leq & \\
  &I^{'} (I_1 (a_1)) \cdot I^{'} (I_2 (a_2)) \leq & \\
  &I^{'}(I_1 (a_1) \cdot I_2 (a_2))&
  \end{array}$
\end{proof}

\begin{defin} Interpretation of subexponential signature

  Let $\Sigma = \langle I, \preceq, W, C, E \rangle$ be a subexponential signature, where $|I| = n$ and
  $\mathcal{S} = \{ \Box_1, \dots, \Box_n \}$ be a set of open modalities on quantale $Q$.
  Subexponential interpretation is a contravariant map $\sigma : I \to \mathcal{S}$ defined as follows:

  $\sigma(s_i) = \begin{cases}
  \Box_i : Q \to Q \text{, s.t.} \forall a \in Q, \Box_i(a) = \{ s \in S_i \: | \: s \leq a\},
  \\ \:\:\:\: \text{if $s_i \notin W \cap C \cap E$} \\
  \Box_i : Q \to Q \text{, s.t.} \forall a \in Q, \Box_i(a) = \{ s \in S_i \: | \: s \leq a, \leq \mathds{1}\},
  \\ \:\:\:\: \text{if $s_i \in W$} \\
  \Box_i : Q \to Q \text{, s.t.} \forall a \in Q, \Box_i(a) = \{ s \in S_i \: | \: s \leq a, \in Z(Q) \},
  \\ \:\:\:\: \text{if $s_i \in E$} \\
  \Box_i : Q \to Q \text{, s.t.} \forall a \in Q, \Box_i(a) = \{ s \in S_i \: | \: s \leq a, \forall b, b \cdot s \vee s \cdot b \leq s \cdot b \cdot s \},
  \\ \:\:\:\: \text{if $s_i \in E$} \\
  \text{otherwise, if $s_i$ belongs to some intersection of subsets, then we combine the relevant conditions } \\
  \end{cases}$
\end{defin}

\begin{defin} Let $Q$ be a quantale, $f : Tp \to Q$ a valuation and $\sigma : I \to \mathcal{S}$ a subexponential interpretation, then
  interpretation is defined inductively:

\begin{center}
$\begin{array}{lll}
& [\![p_i]\!] = f(p_i)&\\
& [\![\mathds{1}]\!] = e & \\
&[\![A \bullet B]\!] = [\![A]\!] \cdot [\![B]\!] & \\
&[\![A \backslash B]\!] = [\![A]\!] \backslash [\![B]\!] & \\
&[\![A / B]\!] = [\![A]\!] / [\![B]\!]& \\
&[\![A \& B]\!] = [\![A]\!] \wedge [\![B]\!]& \\
&[\![A \vee B]\!] = [\![A]\!] \vee [\![B]\!]& \\
&[\![!_{s_i} A]\!] = \sigma(s_i) [\![A]\!]&
\end{array}$
\end{center}
\end{defin}

\begin{theorem}
  $\Gamma \rightarrow A \Rightarrow [\![\Gamma]\!] \leq [\![A]\!]$
\end{theorem}

\begin{proof}
We consider the case with polymodal promotion rule.

\begin{enumerate}

\item Let $!_{s_1} A_1, \dots, !_{s_n} A_n \rightarrow A$ and $\forall i, s \preceq s_i$.

Then $\forall a \in Q, \sigma(s_i)(a) \leq \sigma(s)(a)$.

By IH, $\sigma(s_1)[\![A_1]\!] \cdot \dots \cdot \sigma(s_n) [\![A_n]\!] \leq [\![A]\!]$.

Thus, $\sigma(s)(\sigma(s_1)[\![A_1]\!] \cdot \dots \cdot \sigma(s_n) [\![A_n]\!]) \leq \sigma(s)([\![A]\!])$.

By Lemma 5, $\sigma(s_1)[\![A_1]\!] \cdot \dots \cdot \sigma(s_n) [\![A_n]\!] \leq \sigma(s)(\sigma(s_1)[\![A_1]\!] \cdot \dots \cdot \sigma(s_n) [\![A_n]\!])$.

So, $\sigma(s_1)[\![A_1]\!] \cdot \dots \cdot \sigma(s_n) [\![A_n]\!] \leq \sigma(s)([\![A]\!])$.

\item Let $\Gamma, A, \Delta \rightarrow B$.

By IH, $[\![\Gamma]\!] \cdot [\![A]\!] \cdot [\![\Delta]\!] \leq [\![B]\!]$.

By the definition, $\sigma(s_i)([\![A]\!]) \leq [\![A]\!]$.

So, $[\![\Gamma]\!] \cdot \sigma(s_i)([\![A]\!]) \cdot [\![\Delta]\!] \leq [\![B]\!]$

\item Let $\Gamma, \Delta \rightarrow B$, $A \in Fm$, and $s_i \in W$.

So, $[\![\Gamma]\!] \cdot [\![\Delta]\!] \leq [\![B]\!]$,
then $[\![\Gamma]\!] \cdot e \cdot [\![\Delta]\!] \leq [\![B]\!]$, where $e \in Q$ is unit.

By the definition of unital open modality, $\sigma(s_i)([\![A]\!]) \leq e$.

Thus, $[\![\Gamma]\!] \cdot \sigma(s_i)([\![A]\!]) \cdot [\![\Delta]\!] \leq [\![B]\!]$.

\item Let $\Gamma, !_{s_i} A, \Delta, !_{s_i} A, \Pi \rightarrow B$ and $s_i \in C$.

By IH, $[\![\Gamma]\!] \cdot \sigma(s_i)([\![A]\!]) \cdot [\![\Delta]\!] \cdot \sigma(s_i)([\![A]\!]) \cdot [\![\Pi]\!] \leq [\![B]\!]$.

By the definition, $\sigma(s_i)([\![A]\!]) \cdot [\![\Delta]\!] \leq \sigma(s_i)([\![A]\!]) \cdot [\![\Delta]\!] \cdot \sigma(s_i)([\![A]\!])$.

Then $[\![\Gamma]\!] \cdot \sigma(s_i)([\![A]\!]) \cdot [\![\Delta]\!] \cdot [\![\Pi]\!] \leq [\![B]\!]$

\item Let $\Gamma, !_{s_i} A, \Delta, \Pi \rightarrow B$ and $s_i \in E$, so $\sigma(s_i)(a) \in Z(Q)$ for all $a \in Q$ by the definition.

By IH, $[\![\Gamma]\!] \cdot \sigma(s_i)([\![A]\!]) \cdot [\![\Delta]\!] \cdot [\![\Pi]\!] \leq [\![B]\!]$

Hence, $[\![\Gamma]\!] \cdot [\![\Delta]\!] \cdot \sigma(s_i)([\![A]\!]) \cdot [\![\Pi]\!] \leq [\![B]\!]$.

\end{enumerate}

\end{proof}

\section{Quantale completeness}

\begin{defin}
$ $

  Let $\mathcal{F} \subseteq Fm$, an ideal is a subset $\mathcal{I} \subseteq \mathcal{F}$, such that:

\begin{itemize}
  \item If $B \in \mathcal{I}$ and $A \rightarrow B$, then $A \in \mathcal{I}$;
  \item If $A, B \in \mathcal{I}$, then $A \lor B \in \mathcal{I}$.
\end{itemize}
\end{defin}

\begin{defin}
$ $

  Let $S \subseteq \mathcal{F} \subseteq Fm$,
  then $\bigvee S = \bigcap \{ \mathcal{I} \subseteq \mathcal{F} \: | \: S \subseteq \mathcal{I} \}$
\end{defin}

\begin{prop}
  $\bigvee S$ is an ideal.
\end{prop}

\begin{lemma}
  $A \subseteq Fm$, then $\{ B \: | \: B \rightarrow A' \} = \bigvee A$.
\end{lemma}

\begin{proof}
$ $

Let $A \subseteq Fm$. Then $\{ B \: | \: B \rightarrow A', A' \in A \} \subseteq \bigvee A$, so far as $A' \rightarrow A'$ by axiom.

On the other hand, $\{ B \: | \: B \rightarrow A', A' \in A \}$ is an ideal, hence, $A \subseteq \{ B \: | \: B \rightarrow A', A' \in A \}$.
\end{proof}

\begin{lemma}
  $\bigvee A \subseteq \bigvee B$ iff $\forall A' \in A, \forall B' \in B, A' \rightarrow B'$.
\end{lemma}

\begin{proof}
  Let $\bigvee A \subseteq \bigvee B$,
  then $\{ C | C \rightarrow A', A' \in A \} \subseteq \{ D \: | \: D \rightarrow B', B' \in B \}$.

Thus, for all $A' \in A$, $A' \in \{ C | C \rightarrow A', A' \in A \}$,
then $A' \in \{ D \: | \: D \rightarrow B', B' \in B \}$, hence $A' \rightarrow B'$, for all $B' \in B$.

On the other hand, let $A' \rightarrow B'$ for all $A' \in A$, $B' \in B$ and $C \in \bigvee A$.

Thus, $C \rightarrow A'$, then $C \rightarrow B'$ by cut, so $C \in B'$.

\end{proof}

\begin{lemma}
  Let $\mathcal{Q} = \{ \bigvee S \: | \: S \subseteq Fm \}$ and $\bigvee \mathcal{A} \cdot \bigvee \mathcal{B} =
  \{ A \bullet B \: | \: A \in \mathcal{A}, B \in \mathcal{B} \}$.
  Then $\langle \mathcal{Q}, \subseteq, \cdot, \bigvee{{\bf 1}}\rangle$ is a quantale.
\end{lemma}

\begin{proof}
  See
\end{proof}

\begin{lemma} Interior lemma.
$ $

Let $Q_1 \subseteq \mathcal{Q}$, define a map $\Box : Q \to Q$, such that $\Box(A) = \{  Q \in Q_1 \: | \: Q \subseteq A \}$.
Then $\Box$ is a quantic conucleus.
\end{lemma}

\begin{lemma}
$ $

  Let $A_1, A_2 \subseteq Fm$ and $!_s A_i = \{ !_s W \: | \: W \in S_i \}$, for $i = 1,2$.

  Then $\bigvee (!_s A_1 \cdot !_s A_2) \subseteq \bigvee (!_s (A_1 \cdot A_2))$.
\end{lemma}

\begin{proof}
$ $

$\bigvee (!_s A_1 \cdot !_s A_2) = \bigvee \{ \bigvee W \: | \: \bigvee W \subseteq \bigvee (!_s A_1 \cdot !_s A_2) \}$.

Let $W^{'} \in \bigvee (!_s A_1 \cdot !_s A_2)$,
then $W^{'} \rightarrow \: !_s A_1^{'} \bullet !_s A_1^{'}$ for each $A_i^{'} \in A_i$.
But, $!_s A_1^{'} \bullet !_s A_2^{'} \rightarrow !_s (A_1^{'} \bullet A_2^{'})$.

Then, $W^{'} \rightarrow !_s (A_1^{'} \bullet A_2^{'})$ by cut, then $W^{'} \in \bigvee (!_s (A_1 \cdot A_2))$.
\end{proof}

\begin{lemma}
  Let $!_s \in I$, $I \notin W \cap E \cap C$ and $Q \subseteq \mathcal{Q}$.
  Then there exist a subset $Q \subseteq \mathcal{Q}$ and
  a quantic conucleus $\Box_s (\bigvee \{ A \}) = \{ \bigvee Q \in \mathcal{Q} \: | \: \}$
\end{lemma}

\begin{proof}
\end{proof}

\begin{proof}
  See
\end{proof}

\begin{lemma} Let $Q \subseteq \mathcal{Q}$, then the following operators are quantic conuclei:
  \begin{enumerate}
    \item $\Box_z (A) = \bigvee \{ \bigvee \{ W \} \in Q \: | \bigvee \{ W \} \subseteq \bigvee \{ A \}, \bigvee \{ W \} \in Z(Q) \}$;
    \item $\Box_{{\bf 1}} (A) = \bigvee \{ \bigvee \{ W \} \in Q \: | \bigvee \{ W \} \subseteq \bigvee \{ A \}, \bigvee \{ W \} \subseteq \bigvee \{ {\bf 1} \} \}$;
    \item $\Box_{idem} (A) = \bigvee \{ \bigvee \{ W \} \in Q \: | \bigvee \{ W \} \subseteq \bigvee \{ A \}, \forall B \in Fm, (\bigvee \{ B \} \cdot \bigvee \{ W \}) \cup (\bigvee \{ W \} \cdot \bigvee \{ B \}) \subseteq \bigvee \{ W \} \cdot \bigvee \{ A \} \cdot \bigvee \{ W \} \}$;
    \item $\Box_{z, \mathds{1}}, \Box_{z, idem}, \Box_{\mathds{1}, idem}, \Box_{z, \mathds{1}, idem}$.
  \end{enumerate}
\end{lemma}

\begin{proof}
  Follow from one of lemmas above.
\end{proof}

\begin{lemma}
  Let $!_s \in I$, $I \notin W \cap E \cap C$,
  then $\Box_s (\bigvee A) = \bigvee \{ !_s B \: | \: !_s B \rightarrow \bigvee A^{'}, A^{'} \in A \}$
  is a quantic conucleus.
\end{lemma}

\begin{proof}
$ $

\begin{enumerate}
  \item $\Box_s (\bigvee A) \subseteq \bigvee A$;

$\Box_s (\bigvee A) = \Box_s (\{ B \: | \: B \rightarrow A', A' \in A\}) =
\{ !_s B \: | \: !_s B \rightarrow A', A' \in A\}$.

Let $!_s B \in \Box_s (\bigvee A)$, then $!_s B \rightarrow A'$, $A' \in A$, hence $!_s B \in \bigvee A$.

\item $\Box_s (\Box_s(\bigvee A)) = \bigvee \Box_s (\bigvee A)$;

$\Box_s (\Box_s(\bigvee A)) = \{ !_s !_s B \: | \: !_s !_s F \rightarrow \bigvee A^{'}, A^{'} \in A \}$.

Follows from equivalence $!_s !_s B \leftrightarrow !_s B$.

\item $\bigvee A \subseteq \bigvee B \Rightarrow \Box_s (\bigvee A) \subseteq \Box_s (\bigvee B)$;

Follows from admissiability of K-rule for all $s \in I$.

\item $\Box_s \bigvee A \cdot \Box_s \bigvee B = \Box_s (\Box_s \bigvee A \cdot \Box_s \bigvee B)$.

$\begin{array}{lll}
&\Box_s \bigvee A \cdot \Box_s \bigvee B = & \\
&\bigvee \{ !_s C \bullet !_s D \: | \: !_s C \rightarrow A^{'}, !_s D \rightarrow B^{'}, A^{'} \in A, B^{'} \in B \} = & \\
&\bigvee \{ !_s (!_s C \bullet !_s D) \: | \: !_s C \rightarrow A^{'}, !_s D \rightarrow B^{'}, A^{'} \in A, B^{'} \in B \} = & \\
&\Box_s \bigvee \{ !_s C \bullet !_s D \: | \: !_s C \rightarrow A^{'}, !_s D \rightarrow B^{'}, A^{'} \in A, B^{'} \in B \} = & \\
&\Box_s (\Box_s \bigvee A \cdot \Box_s \bigvee B)&
\end{array}$

\end{enumerate}
\end{proof}

\begin{lemma}
  \begin{enumerate}
    \item Let $s \in W$, then for all $A \subseteq Fm$, ${\bf 1} \in \Box_s (\bigvee A)$;
    \item Let $s \in E$, then $\Box_s (\bigvee A) \cdot \bigvee B = \bigvee B \cdot \Box_s (\bigvee A)$.
    \item Let $s \in C$, then $(\Box_s \bigvee A \cdot \bigvee B) \cup (\bigvee B \cdot \Box_s \bigvee A) \subseteq \Box_s \bigvee A \cdot \bigvee B \cdot \Box_s \bigvee A$, for all $B \subseteq Fm$.
  \end{enumerate}
\end{lemma}

\begin{proof}
  \begin{enumerate}
    \item Let $s \in W$, then for all $A \subseteq Fm$, $\Box_s (\bigvee A) = \{ !_s B \: | \: !_s B \rightarrow A^{'}, A^{'} \in A \}$.
    But, $!_s B \rightarrow {\bf 1}$, hence, $1 \in \Box_s (\bigvee A)$, so far as $\Box_s (\bigvee A)$ is an ideal.
    \item  $ $

    $\begin{array}{lll}
    &\Box_s (\bigvee A) \cdot \bigvee B = & \\
    &\bigvee \{ !_s C \bullet D \: | \: !_s C \bullet D \rightarrow A^{'} \bullet B^{'}, A^{'} \in A, B^{'} \in B \} = & \\
    &\bigvee \{ D \bullet !_s C \: | \: D \bullet !_s C \rightarrow A^{'} \bullet B^{'}, A^{'} \in A, B^{'} \in B \} = & \\
    &\bigvee B \cdot \Box_s (\bigvee A)&
    \end{array}$

    \item $ $

    $\Box_s \bigvee A \cdot \bigvee B = \bigvee \{ !_s C \bullet D \: | \: !_s C \bullet D \rightarrow A^{'} \bullet B^{'}\}$.
    $!_s C \bullet D \rightarrow !_s C \bullet D \bullet !_s C$, hence $\Box_s \bigvee A \cdot \bigvee B \subseteq \Box_s \bigvee A \cdot \bigvee B \cdot \Box_s \bigvee A$.

    Similarly with $\bigvee B \cdot \Box_s \bigvee A$.
  \end{enumerate}
\end{proof}

\begin{lemma}
  Let $Q$ be a quantale constructed above and $\Box_1, \dots, \Box_n$ be a family of quantic conuclei on $Q$.
  Then there exist a model $\langle Q, [\![.]\!]\rangle$, such that $[\![A]\!] = \bigvee \{ A \}$, $A \in Fm$.
\end{lemma}

\begin{proof}
\end{proof}

\begin{theorem}
  $\Gamma \models A \Rightarrow \Gamma \rightarrow A$.
\end{theorem}

\end{document}
